\documentclass{../c-lecture}

\subtitle{Structures}

\begin{document}

\begin{frame}
  \titlepage{}
\end{frame}
\begin{frame}
  \frametitle{Outline}
  \tableofcontents{}
\end{frame}

\section{Introduction}

\begin{frame}
  \frametitle{Introduction}
  \begin{itemize}
    \item Our variables until now
    \begin{itemize}
      \item Single variable
      \mint{c}|int i, char c, float f|
      \item Set of \textit{\color{YellowOrange} same type} elements: Array
      \mint{c}|int a[10], char c[20]|
    \end{itemize}
  \end{itemize}
\end{frame}

\begin{frame}
  \begin{itemize}
    \item
      If data are not same type, but related? Example: Information about
      students:
    \begin{itemize}
      \item Student Name
      \item Student Family Name
      \item Student Number
      \item Student Grade
    \end{itemize}
  \end{itemize}
\end{frame}

\begin{frame}[fragile]
  \frametitle{Introduction}
  \begin{itemize}
    \item How to save the student information?
    \item 1- Use separated variables
    \begin{minted}[bgcolor=Black]{c}
char st_name[20];
char st_fam_name[20];
int id;
int grade;
    \end{minted}
    \item 2- Put them altogether, they are related
    \begin{itemize}
      \item Use \textit{\color{Cyan} struct}
      \item
        This concept is extended in OOP as the
        <span class="hl-green">object</span>
    \end{itemize}
  \end{itemize}
\end{frame}

\section{Struct Definition}

\begin{frame}
  \frametitle{struct: version 1}
  \begin{itemize}
    \item Set of related variables
    \begin{itemize}
      \item
        Each variable in <span class="hl-cyan">struct</span> has its
        <span class="hl-orange">own type</span>

      \item <span class="hl-cyan">struct</span> in C (version 1)
      <p>
        <span class="hl-cyan">struct</span> {<br />
        &nbsp;<span class="hl-violet">&lt;variable declaration&gt;</span><br />
        } <span class="hl-green">&lt;identifier list&gt;</span>;<br />
      </p>
    \end{itemize}
  \end{itemize}
\end{frame}

\begin{frame}[fragile]
  \frametitle{struct (version 1): Example}
  \begin{minted}[bgcolor=Black]{c}
struct{
  char st_name[20];
  char st_fam_name[20];
  int id;
  int grade;
} st1;
  \end{minted}
  \begin{itemize}
    \item We declare a variable <span class="hl-orange">st1</span>
    \item Type of <span class="hl-orange">st1</span> is struct
    \item
      <span class="hl-violet">id</span> is a
      <span class="hl-green">member</span> of the struct

    \item
      <span class="hl-violet">grade</span> is a
      <span class="hl-green">member</span> of the struct

  \end{itemize}
\end{frame}

\begin{frame}
  \frametitle{struct (version 1): Example}
  <pre><code class="hljs lang-c">
struct{
  char st_name[20];
  char st_fam_name[20];
  int id;
  int grade;
} st1, st2, st3;
  </code></pre>
  \begin{itemize}
    \item
      We declare three variables: <span class="hl-orange">st1</span>,
      <span class="hl-orange">st2</span>,
      <span class="hl-orange">st3</span>

    \item
      Type of <span class="hl-orange">st1</span>,
      <span class="hl-orange">st2</span>, <span class="hl-orange">st3</span> is
      the struct

    \item
      In this model, we cannot reuse the struct definition in other location
      (e.g., input of function)

  \end{itemize}
\end{frame}
\begin{frame}
  \frametitle{struct: Version 2}
  \begin{itemize}
    \item <span class="hl-cyan">struct</span> in C (version 2)
    <p>
      <span class="hl-cyan">struct</span>
      <span class="hl-yellow">&lt;tag&gt;</span> {<br />
      &nbsp;<span class="hl-violet">&lt;variable declaration&gt;</span><br />
      };<br />
      <span class="hl-cyan">struct</span>
      <span class="hl-yellow">&lt;tag&gt;</span>
      <span class="hl-green">&lt;identifier list&gt;</span>;<br />
    </p>
  \end{itemize}
\end{frame}
\begin{frame}
  \frametitle{struct (version 2): Example}
  <pre><code class="hljs lang-c">
struct std_info {
  char st_name[20];
  char st_fam_name[20];
  int id;
  int grade;
};
struct std_info st1, st2, st3;
  </code></pre>
  \begin{itemize}
    \item We define a struct with tag <span class="hl-orange">std_info</span>
    \begin{itemize}
      \item We don’t allocate memory, it is just definition
    \end{itemize}
    \item
      We declare variables <span class="hl-green">st1</span>,
      <span class="hl-green">st2</span>, <span class="hl-green">st3</span> from
      <span class="hl-orange">std_info</span>

  \end{itemize}
\end{frame}
\begin{frame}
  \frametitle{typedef}
  \begin{itemize}
    \item We can assign a new name for each type
    \begin{itemize}
      \item
        Assign name <span class="hl-green">“integer”</span> to
        <span class="hl-orange">“int”</span>

      \item
        Assign name <span class="hl-green">“int_array”</span> to
        <span class="hl-orange">“int[100]”</span>

      \item
        Assign name <span class="hl-green">“int_pointer”</span> to
        <span class="hl-orange">“int *”</span>

    \end{itemize}
    \item New names are assigned by <span class="hl-violet">typedef</span>
    \item
      After we assigned the new name, we can use it in identifier declaration

  \end{itemize}
\end{frame}
\begin{frame}
  \frametitle{typedef: Examples}
  <pre><code class="hljs lang-c">
/* Assign new name integer to type int */
typedef int integer;

/* Use the new name */
integer i, j, k;

/* Assign new name alephba to type char */
typedef char alephba;

/* Use the new name */
alephba c1, c2;
  </code></pre>
\end{frame}
\begin{frame}
  \frametitle{typedef: Examples}
  <pre><code class="hljs lang-c">
/* Assign new name intptr to type int * */
typedef int * intptr;

/* Use the new name */
intptr pi, pj, pk;

typedef int int_arr1[10], int_arr2[20];

int_arr1 array1;
int_arr2 array2;
  </code></pre>
\end{frame}
\begin{frame}
  \frametitle{typedef: Cons}
  <pre><code class="hljs lang-c">
typedef int arr[100];

int main() {
  arr a1;
  arr a2;

  a1 = a2; // error: array type 'arr' (aka 'int [100]') is not assignable
}
  </code></pre>
\end{frame}
\begin{frame}
  \frametitle{struct: Version 3.1}
  \begin{itemize}
    \item <span class="hl-cyan">struct</span> in C (version 3.1)
    \item Using the <span class="hl-orange">typedef</span>
    <p>
      <span class="hl-cyan">struct</span>
      <span class="hl-yellow">&lt;tag&gt;</span> {<br />
      &nbsp;<span class="hl-violet">&lt;variable declaration&gt;</span><br />
      };<br />
      <span class="hl-orange">typedef</span>
      <span class="hl-cyan">struct</span>
      <span class="hl-yellow">&lt;tag&gt;</span>
      <span class="hl-red">&lt;new_name&gt;</span>;<br />
      <span class="hl-red">&lt;new_name&gt;</span>
      <span class="hl-green">&lt;identifier list&gt;</span>;<br />
    </p>
  \end{itemize}
\end{frame}
\begin{frame}
  \frametitle{struct (Version 3.1): Examples}
  <pre><code class="hljs lang-c">
struct std_info{
  char st_name[20];
  char st_fam_name[20];
  int id;
  int grade;
};

typedef struct std_info information;

information st1, st2;
  </code></pre>
\end{frame}
\begin{frame}
  \frametitle{struct: Version 3.2}
  \begin{itemize}
    \item <span class="hl-cyan">struct</span> in C (version 3.2)
    \item Using the <span class="hl-orange">typedef</span>
    <p>
      <span class="hl-orange">typedef</span>
      <span class="hl-cyan">struct</span> {<br />
      &nbsp;<span class="hl-violet">&lt;variable declaration&gt;</span><br />
      } <span class="hl-red">&lt;new_name&gt;</span>;<br />
      <span class="hl-red">&lt;new_name&gt;</span>
      <span class="hl-green">&lt;identifier list&gt;</span>;<br />
    </p>
  \end{itemize}
\end{frame}
\begin{frame}
  \frametitle{struct (Version 3.2): Examples}
  <pre><code class="hljs lang-c">
typedef struct {
  char st_name[20];
  char st_fam_name[20];
  int id;
  int grade;
} information;
information st1, st2;
  </code></pre>
\end{frame}
\begin{frame}
  \frametitle{Structures as New Data Type}
  \begin{itemize}
    \item
      When we define a new struct, in fact we are defining a new data type

    \begin{itemize}
      \item Then we use the new data type and define variables
    \end{itemize}
    \item So, we need to learn how to work it
    \begin{itemize}
      \item Access to members
      \item Operators for struct
      \item Array of struct
      \item struct in functions
      \item Pointer to struct
    \end{itemize}
  \end{itemize}
\end{frame}
\begin{frame}
  \frametitle{Size of struct}
  \begin{itemize}
    \item
      The size of struct is <span class="hl-red">NOT</span> the sum of size of
      members

    \begin{itemize}
      \item
        <pre><code class="hljs lang-c">struct test_size{char c; int i; };</code></pre>

      \item
        <pre><code class="hljs lang-c">sizeof(struct test_size) = 8</code></pre>

    \end{itemize}
    \item This is because of <span class="hl-orange">Structure Padding</span>
    \begin{itemize}
      \item Computer HW cannot (should not) read any arbitrary address
      \item The address should be aligned in word
      \begin{itemize}
        \item 4 bytes in 32-bit machine
      \end{itemize}
      \item The padding is to align the address
      \item
        More details &amp; examples:
        <a href="https://fresh2refresh.com/c-programming/c-structure-padding"
          >Here</a
        >

    \end{itemize}
  \end{itemize}
\end{frame}
\begin{frame}
  <div class="toc" data-selected="2"></div>
\end{frame}
\begin{frame}
  \frametitle{Using struct}
  \begin{itemize}
    \item We should declare variables from struct type
    \begin{itemize}
      \item Versions 1, 2, 3.1, 3.2
    \end{itemize}
    \item How to access to the members of struct
    \begin{itemize}
      \item
        <span class="hl-orange">&lt;struct variable&gt;</span>.<span
          class="hl-green"
          >&lt;element name&gt;</span
        >

      \item
        <span class="hl-cyan">st1.st_name</span> is a array of char in struct
        st1

      \item
        <span class="hl-cyan">st2.grade</span> is a int variable in struct st2

    \end{itemize}
  \end{itemize}
\end{frame}
\begin{frame}
  \begin{frame}
    \begin{itemize}
      \item Similar to array initialization
      <pre><code class="hljs lang-c">
  struct std_info st1 = {"Parham", "Alvani", 9231058, 18};
      </code></pre>
      \item
        <span class="hl-orange">“Parham”</span> is assigned to
        <span class="hl-green">st_name</span>

      \item
        <span class="hl-orange">“Alvani”</span> is assigned to
        <span class="hl-green">st_fam_name</span>

      \item
        <span class="hl-orange">9231058</span> is assigned to
        <span class="hl-green">id</span>

      \item
        <span class="hl-orange">18</span> is assigned to
        <span class="hl-green">grade</span>

      <li class="hl-cyan">
        Order of values should be exactly the order of the members

      <li class="hl-violet">
        The number of values should be &lt;= the number of members

      <li class="hl-red">
        Initial values cannot be assigned in struct definition

    \end{itemize}
  \end{frame}
  \begin{frame}
    <pre><code class="hljs lang-c">
  struct std_info st1 = { .st_name = "Parham", .st_fam_name = "Alvani", .id = 9231058, .score = 18};
    </code></pre>
  \end{frame}
\end{frame}
\begin{frame}
  \frametitle{Example}
  <pre><code class="hljs lang-c">
#include &lt;stdio.h&gt;

typedef struct {
  char name[20];
  char fam_name[20];
  int id;
  int grade;
} information;

void main(void){
  information st2, st1 = {"Parham", "Alvani", 9231058, 20};

  printf("After init: \n");
  printf("Name = %s, \nFam. Name = %s, \nid = %d, \ngrade = %d\n", st1.name, st1.fam_name, st1.id, st1.grade);

  scanf("%s", st2.name);
  scanf("%s", st2.fam_name);
  scanf("%d", &amp;st2.id);
  scanf("%d", &amp;st2.grade);

  printf("Your Input is: \n");
  printf("Name = %s, \nFam. Name = %s, \nid = %d, \ngrade = %d\n", st2.name, st2.fam_name, st2.id, st2.grade);
}
  </code></pre>
\end{frame}
\begin{frame}
  \frametitle{Nested struct}
  <pre><code class="hljs lang-c">
struct date_type{
  int day, month, year;
};

typedef struct{
  char name[20];
  char fam_name[20];
  int id;
  int grade;
  struct date_type date;
} information;
  </code></pre>
\end{frame}
\begin{frame}
  \begin{frame}
    \frametitle{Nested struct}
    <pre><code class="hljs lang-c">
information st1 = {"A", "B", 1, 10, {2, 3, 1368}};
    </code></pre>
    <pre><code class="hljs lang-c">
information st2;
// st2.name = "C"; char[20] is not assignable
// st2.fam_name = "D"; char[20] is not assignable
st2.id = 2;
st2.grade = 15;
    </code></pre>
    <pre><code class="hljs lang-c">
st2.date.day = 10;
st2.date.month = 5;
st2.date.year = 1390;
    </code></pre>
  \end{frame}
  \begin{frame}
    \frametitle{More on "char[20] is not assignable"}
    <pre><code class="hljs lang-c">
struct person {
  char name[200];
  char family[200];
};
    </code></pre>
    <pre><code class="hljs lang-c">
char hello[200] = "Hello world";
printf("%s\n", hello);

hello[0] = 'h';
printf("%s\n", hello);

// hello = "123";

struct person p1 = {"Parham", "Alvani"};
printf("%s\n", p1.name);

p1.name[0] = 'p';
printf("%s\n", p1.name);

// p1.name = "123";
    </code></pre>
  \end{frame}
\end{frame}
\begin{frame}
  \frametitle{struct: Copy and Assignment}
  <pre><code class="hljs lang-c">
struct date_type{
  int day, month, year;
};

struct date_type d1, d2 = {2, 1, 1360};
d1 = d2;

/*
  d1.day = d2.day;
  d1.month = d2.month;
  d1.year = d2.year;
*/
  </code></pre>
\end{frame}
\begin{frame}
  \frametitle{struct: Copy and Assignment 😮}
  <pre><code class="hljs lang-c">
struct test_type{
  char name[10];
  int id[10];
};

struct test_type d1, d2 = {"ABC", {1, 2, 3}};
d1 = d2;
/*
 d1.name = "ABC";
 d1.id = {1, 2, 3};
*/

  </code></pre>
\end{frame}
\begin{frame}
  \frametitle{struct: Comparing}
  \begin{itemize}
    \item
      We <span class="hl-orange">cannot</span> compare
      <span class="hl-cyan">struct</span> variables

    \begin{itemize}
      \item ==, &lt;=, &lt;, &gt;, &gt;= cannot be used for struct
    \end{itemize}
    <pre><code class="hljs lang-c">
information st1, st2;

if(st1 &lt;= st2){
  // Compile Error ...
}

    </code></pre>
    \item Why?
    \begin{itemize}
      \item What does this mean? st1 &lt;= st2
    \end{itemize}
  \end{itemize}
\end{frame}
\begin{frame}
  \frametitle{struct: Comparing}
  \begin{itemize}
    \item We can compare members of structs
    <pre><code class="hljs lang-c">
if(
  (st1.id == st2.id) &amp;&amp;
  (strcmp(st1.name, st2.name) == 0) &amp;&amp;
  (strcmp(st1.fam_name, st2.fam_name) == 0)
) {
  /* st1 == st2 */
}
    </code></pre>
    \item We can define &lt;, &lt;=, &gt;, &gt;= for struct
    <pre><code class="hljs lang-c">
if (
  (st1.id &gt; st2.id) &amp;&amp;
  (strcmp(st1.name, st2.name) == 0) &amp;&amp;
  (strcmp(st1.fam_name, st2.fam_name) == 0)
) {
  /* st1 &gt; st2 */
}
    </code></pre>
  \end{itemize}
\end{frame}
\begin{frame}
  \frametitle{struct: Arithmetic operations}
  \begin{itemize}
    \item No arithmetic operation (+, -, /, ...) is defined for structures
    \item We can define ours operations
    \item We have an example in the following slides
  \end{itemize}
\end{frame}
\begin{frame}
  <div class="toc" data-selected="2"></div>
\end{frame}
\begin{frame}
  \frametitle{Array of struct: Definition}
  \begin{itemize}
    \item
      <span class="hl-cyan">struct</span> is a type &rarr; We can define array
      of struct

    <pre><code class="hljs lang-c">
struct std1{
  int id;
  int grad;
};
struct std1 std_arr[20];

typedef struct{
  int id;
  int grad;
} std2;
std2 std_arr[20];
    </code></pre>
  \end{itemize}
\end{frame}
\begin{frame}
  \frametitle{Upper Average Students 🤓}
  <pre><code class="hljs lang-c">
#include &lt;stdio.h&gt;

int main(void){
  struct std {
    int id;
    int grade;
  };

  const int num = 25;

  double sum, average;
  int i;
  struct std std_arr[num];
  for (i = 0; i &lt; num; i++){
    printf("Enter ID and grade\n");
    scanf("%d", &amp;(std_arr[i].id));
    scanf("%d", &amp;(std_arr[i].grade));
  }

  sum = 0;
  for(i = 0; i &lt; num; i++)
    sum += std_arr[i].grade;

  average = sum / num;

  for (i = 0; i &lt; num; i++)
    if (std_arr[i].grade &gt;= average)
      printf("Student %d passed\n", std_arr[i].id);

  return 0;
}
  </code></pre>
\end{frame}
\begin{frame}
  \frametitle{Find Student}
  <pre><code class="hljs lang-c">
#include &lt;stdio.h&gt;

int main(void) {
  struct std{
    char name[20];
    int id;
    int grade;
  };

  const int num = 25;
  struct std std_arr[num];

  int sid, i;
  for (i = 0; i &lt; num; i++){
    printf("Enter Name, ID and grade\n");
    scanf("%s", std_arr[i].name);
    scanf("%d", &amp;(std_arr[i].id));
    scanf("%d", &amp;(std_arr[i].grade));
  }

  printf("Enter Search ID: ");
  scanf("%d", &amp;sid);

  for (i = 0; i &lt; num; i++) {
    if (std_arr[i].id == sid) {
      printf("Found:\n");
      printf("Name = %s\n", std_arr[i].name);
      printf("ID = %d\n", std_arr[i].id);
      printf("Grade = %s\n", std_arr[i].grade);
    }
  }
  return 0;
}
  </code></pre>
\end{frame}
\begin{frame}
  <div class="toc" data-selected="2"></div>
\end{frame}
\begin{frame}
  \frametitle{Pointer to struct: Definition}
  \begin{itemize}
    \item
      A variable of struct type is a <span class="hl-orange">variable</span>

    \item
      It has <span class="hl-green">address</span>, we can have
      <span class="hl-cyan">pointer</span> to it

    <pre><code class="hljs lang-c">
struct std{
  int id;
  int grade;
};
struct std st1;
struct std *ps;
ps = &st1;
    </code></pre>
  \end{itemize}
\end{frame}
\begin{frame}
  \frametitle{Pointer to struct: Usage (Version 1)}
  \begin{itemize}
    \item We can use *pointer method
    \item
      <span class="hl-orange">*ps</span> means the content of the address that
      <span class="hl-orange">ps</span> refers to there &rarr; it is struct

    \item
      <span class="hl-cyan">(*ps).id</span> is the member of struct that
      <span class="hl-orange">ps</span> refers to it

    \item
      <span class="hl-cyan">(*ps).grade</span> is the member of struct that
      <span class="hl-orange">ps</span> refers to it

  \end{itemize}
\end{frame}
\begin{frame}
  \frametitle{Pointer to struct: Usage (Version 2)}
  \begin{itemize}
    \item We can use <span class="hl-orange">“-&gt;"</span> method
    <pre><code class="hljs lang-c">
struct std {
  int id;
  int grade;
};
struct std st1, *ps;
ps = &amp;st1
// (*ps).id
int y = ps-&gt;id;
// (*ps).grade
int z = ps-&gt;grade;
    </code></pre>
  \end{itemize}
\end{frame}
\begin{frame}
  <div class="toc" data-selected="2"></div>
\end{frame}
\begin{frame}
  \frametitle{struct &amp; Functions}
  \begin{itemize}
    \item struct is a type &rarr; It can be used
    \begin{itemize}
      \item
        In <span class="hl-orange">input</span> parameter list of functions

      \begin{itemize}
        \item Call by <span class="hl-green">value</span>
        \item Call by <span class="hl-green">reference</span>
      \end{itemize}
      \item In <span class="hl-orange">return type</span> of functions
    \end{itemize}
    <pre><code class="hljs lang-c">
void f(struct std s1); // call by value
void g(struct std *s2); // call by reference
struct std h(void); // return type
    </code></pre>
  \end{itemize}
\end{frame}
\begin{frame}
  \frametitle{struct &amp; Functions: Example}
  \begin{itemize}
    \item struct as call by value input parameter
  \end{itemize}
  <pre><code class="hljs lang-c">
void print_st_info(information st){
  printf("Name = %s\n", st.name);
  printf("Fam = %s\n", st.fam_name);
  printf("id = %d\n", st.id);
  printf("grade = %d\n", st.grade);
}
// ----- Calling the function ----
information st1;
print_st_info(st1);
  </code></pre>
\end{frame}
\begin{frame}
  \frametitle{struct &amp; Functions: Example}
  \begin{itemize}
    \item struct as call by reference input parameter
  \end{itemize}
  <pre><code class="hljs lang-c">
void read_st_info(information *pst) {
  scanf("%s", pst-&gt;name);
  scanf("%s", pst-&gt;fam_name);
  scanf("%d", &amp;(pst-&gt;id));
  scanf("%d", &amp;(pst-&gt;grade));
}
// ----- Calling the function ----
information st1;
read_st_info(&amp;st1);
  </code></pre>
\end{frame}
\begin{frame}
  \begin{frame}
    \frametitle{struct &amp; Functions in Action}
  \end{frame}
  \begin{frame}
    <pre><code class="hljs lang-c">
struct student {
  char first_name[200];
  char last_name[200];
  int id;
};
    </code></pre>
  \end{frame}
  \begin{frame}
    <p>
      Sets ID field of student with given identification. As you may guess this
      function must recieve <span class="hl-orange">a pointer</span> of student
      structure.
    </p>
    <pre><code class="hljs lang-c">
void student_set_id(struct student *std, int id) {
  std-&gt;id = id;
}
    </code></pre>
  \end{frame}
  \begin{frame}
    <p>
      Sets first name field of student with given Name As you may guess this
      function must recieve <span class="hl-orange">a pointer</span> of student
      structure. But why?
    </p>
    <pre><code class="hljs lang-c">
void student_set_id(struct student *std, const char *name) {
  strcpy(sed-&gt;first_name, name)
}
    </code></pre>
  \end{frame}
  \begin{frame}
    <p>
      First name field of stduent is an array and as you may remember arrays
      name contains the address of array's first cell. So is the following code
      correct? 🤔
    </p>
    <pre><code class="hljs lang-c">
void student_set_id_wrong(struct student std, const char *name) {
  strcpy(sed.first_name, name)
}
    </code></pre>
    <p>
      <span class="hl-red">No.</span> It doesn't change first name of given
      student structure.
    </p>
    <p>
      Because C copies all of the structures' members,
      <span class="hl-green">even if they are arrays</span>, so
      <span class="hl-orange">Calling by Value</span> makes a copy from each
      field of the structure that changing them doesn't change anything in the
      source.
    </p>
  \end{frame}
  \begin{frame}
    <p>
      As a rule of thumb <span class="hl-violet">always</span> use pointer of
      structures in your functions even you don't want to change them. In case
      of read-only functions use <span class="hl-cyan">const</span> pointers to
      tell the users you don't want to change the given structure.
    </p>
    <pre><code class="hljs lang-c">
void student_print(const struct student *std) {
  printf("Name: %s %s\n", std-&gt;first_name, std-&gt;last_name);
  printf("ID: %d\n", std-&gt;id);
}
    </code></pre>
  \end{frame}
\end{frame}
\begin{frame}
  \frametitle{struct &amp; Functions: Example}
  \begin{itemize}
    \item struct as output of function
  \end{itemize}
  <pre><code class="hljs lang-c">
information create_st_info(void) {
  information tmp;
  scanf("%s", tmp.name);
  scanf("%s", tmp.fam_name);
  scanf("%d", &amp;tmp.id);
  scanf("%d", &amp;tmp.grade);
  return tmp;
}
// ----- Calling the function ----
information st1;
st1 = create_st_info();
  </code></pre>
\end{frame}
\begin{frame}
  \frametitle{Scope of struct definition}
  \begin{itemize}
    \item A struct can be used only
    \begin{itemize}
      \item In the defined scope
      \item After definition
    \end{itemize}
    \item if sruct is defined in a function
    \begin{itemize}
      \item It can be used only in the function
      \item No other function knows about it
    \end{itemize}
    \item If struct is defined as a global
    \begin{itemize}
      \item It can be used in all function after the definition
    \end{itemize}
  \end{itemize}
\end{frame}
\begin{frame}
  \frametitle{Scope of struct variables}
  \begin{itemize}
    \item
      The scope of struct <span class="hl-orange">variables</span> are the same
      other variables

    \item If struct variable is global
    \begin{itemize}
      \item
        Initialized to zero and visible to the functions after its declaration

    \end{itemize}
    \item If struct variable is automatic local
    \begin{itemize}
      \item There is not any initial value, destroyed when the block finishes
    \end{itemize}
    \item If struct variable is static
    \begin{itemize}
      \item Kept in memory until program finishs
    \end{itemize}
  \end{itemize}
\end{frame}
\begin{frame}
  <div class="toc" data-selected="3"></div>
\end{frame}
\begin{frame}
  \begin{frame}
    \frametitle{More Dynamic Data Structures}
    \begin{itemize}
      \item In Arrays
      \begin{itemize}
        \item
          We know the size of array when you
          <span class="hl-green">develop code (coding time)</span>

        \item
          We know the size of array when
          <span class="hl-cyan">program runs</span>

      \end{itemize}
      \item
        What can we do, if we
        <span class="hl-orange">don’t know data size</span> even in run time?

      \begin{itemize}
        \item We use dynamic memory allocation &amp; resize
        \begin{itemize}
          \item Resizing array has cost &amp; overhead
        \end{itemize}
      \end{itemize}
    \end{itemize}
  \end{frame}
  \begin{frame}
    \begin{itemize}
      \item
        What can we do, if we want to
        <span class="hl-orange">add/remove</span> an element to/from middle of
        the array?

      \begin{itemize}
        \item We use dynamic memory allocation &amp; resize
        \begin{itemize}
          \item Resizing array has cost &amp; overhead
        \end{itemize}
        \item Is there any other better approach?
      \end{itemize}
    \end{itemize}
  \end{frame}
\end{frame}
\begin{frame}
  \frametitle{Dynamic Data Structures: Linked List}
  \begin{itemize}
    \item
      <span class="hl-orange">linked list</span> data structure can be used to
      implement the dynamic structures

    \item linked list: Nodes that linked together
    \begin{itemize}
      \item <span class="hl-green">info(s)</span>: Save the information
      \item <span class="hl-cyan">next</span>: Pointer to the next node
      \item
        <span class="hl-violet">previous</span>: Pointer to the previous node

    \end{itemize}
  \end{itemize}
\end{frame}
\begin{frame}
  <img alt="ll-1" src="img/ll-1.png" />
  <img alt="ll-2" src="img/ll-2.png" />
\end{frame}
\begin{frame}
  \frametitle{linked list in C}
  \begin{itemize}
    \item
      linked list is implemented by
      <span class="hl-orange">struct</span> and
      <span class="hl-green">pointer to struct</span>

    \item Struct has a member to save the info
    \item Struct has a pointer to point the next node
    <pre><code class="hljs lang-c">
struct node{
  int info;
  struct node *next;
};
    </code></pre>
  \end{itemize}
\end{frame}
\begin{frame}
  <p>
    Please note that we <span class="hl-red">cannot</span> have structures that
    <span class="hl-green">contain themselves</span>:
  </p>
  <pre><code class="hljs lang-c">
struct node{
  int info;
  struct node next;
}; // compile error
  </code></pre>
  <p>
    <span class="hl-orange">struct node</span> conatins a pointer to its type.
  </p>
\end{frame}
\begin{frame}
  \frametitle{Create nodes}
  \begin{itemize}
    \item We need a function to create each node in list. The function does:
    <ol>
      \item <span class="hl-orange">Allocate</span> the memory
      \item Set the <span class="hl-green">info</span> member
      \item Set the <span class="hl-cyan">next</span> member
      \item <span class="hl-violet">Return</span> the pointer to new node
    </ol>
  \end{itemize}
\end{frame}
\begin{frame}
  \frametitle{Create Node}
  <pre><code class="hljs lang-c">
struct node{
  int info;
  struct node *next;
};


// Returning pointer!?!
// Is it safe?
// Why?
struct node* node_construct(int i) {
  struct node* nn = NULL;

  nn = malloc(sizeof(struct node));
  if (nn == NULL) {
   return NULL;
  }

  nn-&gt;info = i;
  nn-&gt;next = NULL;

  return nn;
}
  </code></pre>
\end{frame}
\begin{frame}
  <p>
    We <span class="hl-orange">can return</span> the address of
    <span class="hl-green">heap-allocated</span> variables
    <span class="hl-violet">from functions</span>. Heap variables are under our
    control, so they are <span class="hl-cyan">continue existing</span> after
    the function completes.
  </p>
\end{frame}
\begin{frame}
  \frametitle{Example: 3 Nodes List}
  <pre><code class="hljs lang-c">
struct node* list = NULL;
list = node_construct(10);
list-&gt;next = node_construct(20);
list-&gt;next-&gt;next = node_construct(30);
  </code></pre>
\end{frame}
\begin{frame}
  \frametitle{Operation on linked list}
  \begin{itemize}
    \item Print the list: <span class="hl-orange">list_print</span>
    \item
      Add new node to end of list:
      <span class="hl-green">list_push_back</span>

    \item
      Add new node to front of list:
      <span class="hl-yellow">list_push_front</span>

    \item
      Insert new node after some node:
      <span class="hl-cyan">list_push_next</span>

    \item
      Delete the first node in list:
      <span class="hl-violet">list_pop_front</span>

    \item
      Delete the end node in list:
      <span class="hl-material">list_pop_back</span>

    \item
      Delete a node from the middle of list:
      <span class="hl-red">list_pop_next</span>

  \end{itemize}
\end{frame}
\begin{frame}
  \frametitle{list_print}
  <pre><code class="hljs lang-c">
void list_print(struct node *head){
  struct node* current;

  for(current = head; current != NULL; current = current-&gt; next) {
    printf("%d\n", current-&gt;info)
  }
}
  </code></pre>
\end{frame}
\begin{frame}
  \frametitle{list_push_back}
  <pre><code class="hljs lang-c">
void list_push_back(struct node* head, struct node* nn){
  struct node* current;

  for(current = head; current-&gt; next != NULL; current = current-&gt; next);

  current-&gt;next = nn;
  nn-&gt;next = NULL;
}
  </code></pre>
\end{frame}
\begin{frame}
  \begin{frame}
    \frametitle{list_pop_back (if more than 1 nodes)}
    <p>X &rarr; Y &rarr; NULL</p>
    <hr />
    \begin{itemize}
      \item Y = X-&gt;next
    \end{itemize}
    <hr />
    <ol>
      \item free(Y)
      \item X-&gt;next = NULL
    </ol>
  \end{frame}
  \begin{frame}
    <pre><code class="hljs lang-c">
void list_pop_back(struct node* head) {
  struct node* current = head;
  while(current-&gt;next-&gt;next != NULL) {
    current = current-&gt;next;
  }

  free(current-&gt;next);
  current-&gt;next = NULL;
}
    </code></pre>
  \end{frame}
\end{frame}
\begin{frame}
  \begin{frame}
    \frametitle{list_push_front 🤔}
    <pre><code class="hljs lang-c">
void list_push_front(struct node* head, struct node* nn) {
  nn-&gt;next = list;
  head = nn;
}
    </code></pre>
  \end{frame}
  \begin{frame}
    \frametitle{list_push_front 😱}
    <pre><code class="hljs lang-c">
void list_push_front(struct node** phead, struct node* nn) {
  nn-&gt;next = *phead;
  *phead = nn;
}
    </code></pre>
  \end{frame}
\end{frame}
\begin{frame}
  \begin{frame}
    \frametitle{list_pop_next}
    <p>X &rarr; Y &rarr; Z</p>
    <hr />
    \begin{itemize}
      \item Y = X-&gt;next
      \item Z = Y-&gt;next
      \item Z = X-&gt;next-&gt;next
    \end{itemize}
    <hr />
    <ol>
      \item X-&gt;next = Z
      \item free(Y)
    </ol>
  \end{frame}
  \begin{frame}
    <pre><code class="hljs lang-c">
void list_pop_by_info(struct node* head, int v) {
  // find a node before the node that contains v
  struct node* current = head;
  for (current = list; current-&gt;next != NULL &amp;&amp; current-&gt;next-&gt;info != v; current = current-&gt;next);

  // there is no node that conaints v
  if (current-&gt;next == NULL) {
    return;
  }

  list_pop_next(current);
}
    </code></pre>
    <pre><code class="hljs lang-c">
void list_pop_next(struct node* adjacent_to_delete_node) {
  struct node* to_delete = adjacent_to_delete_node-&gt;next;
  adjacent_to_delete_node-&gt;next = to_delete-&gt;next;
  free(to_delete);
}
    </code></pre>
  \end{frame}
\end{frame}
\begin{frame}
  \frametitle{Common Bugs}
  \begin{itemize}
    \item
      The last <span class="hl-orange">NULL</span> in Liked-list is very
      important

    \begin{itemize}
      \item Always keep it
    \end{itemize}
    \item Operation of linked-list has many exceptions
    \begin{itemize}
      \item When list is empty
      \item When we want to add to the first of list
      \item ...
    \end{itemize}
  \end{itemize}
\end{frame}
\begin{frame}
  <div class="toc" data-selected="4"></div>
\end{frame}
\begin{frame}
  \frametitle{Introduction}
  \begin{itemize}
    \item Some data are naturally ordered
    \begin{itemize}
      \item Days of week
      \item Months of year
    \end{itemize}
    \item We want to use the order, e.g.
    \begin{itemize}
      \item The number of visitors per day
      \item The salary per month
    \end{itemize}
    \item We need an array
    \begin{itemize}
      \item
        visitors[<span class="hl-green">0</span>] &rarr; The number of visitors
        in <span class="hl-orange">Saturday</span>

      \item
        visitors[<span class="hl-green">1</span>] &rarr; The number of visitors
        in <span class="hl-orange">Sunday</span>

    \end{itemize}
  \end{itemize}
\end{frame}
\begin{frame}
  \frametitle{Introduction}
  \begin{itemize}
    \item
      <span class="hl-orange">Enumeration</span> is a mechanism to assign a name
      for each number

    \item We can use numbers instead of numbers
    \begin{itemize}
      \item More readable core
    \end{itemize}
    \item E.g.
    <pre><code class="hljs lang-c">
visitors[saturday], visitors[friday]
salary[april], salary[june]
    </code></pre>
  \end{itemize}
\end{frame}
\begin{frame}
  \frametitle{enum}
  \begin{itemize}
    \item
      <span class="hl-orange">enum</span> is used to define a set of names and
      their corresponding numbers

    <pre><code class="hljs lang-c">
enum tag {name_1, name_2, …, name_N};
    </code></pre>
    \item <span class="hl-green">tag</span> is the enumeration type
    \begin{itemize}
      \item We use it to define variables
    \end{itemize}
    \item
      <span class="hl-cyan">name_1</span> is
      <span class="hl-red">0</span>

    \item
      <span class="hl-cyan">name_2</span> is
      <span class="hl-red">1</span>

    \item
      <span class="hl-cyan">name_{i}</span> is
      <span class="hl-red">name_{i-1} + 1</span>

  \end{itemize}
\end{frame}
\begin{frame}
  \frametitle{enum}
  <pre><code class="hljs lang-c">
enum week {sat, sun, mon, tue, wed, thu, fri};
  </code></pre>
  <p>sat = 0, sun = 1, mon = 2, …, fri = 6</p>
  <pre><code class="hljs lang-c">
enum year {feb, jan, mar, apr, may, jun, jul, aug, sep, oct, nov, des};
  </code></pre>
  <p>feb = 0, jan = 1, …, nov = 10, des = 11</p>
\end{frame}
\begin{frame}
  \frametitle{enum}
  <p>We can assign the numbers</p>
  <pre><code class="hljs lang-c">
enum week {sat = 1, sun, mon, tue, wed, thu, fri};
  </code></pre>
  <p>sat = 1, sun = 2, mon = 3, …, fri = 7</p>
  <pre><code class="hljs lang-c">
enum condition {False = 0, True, No = 0, Yes, Ghalat = 0, Dorost};
  </code></pre>
  <p>
    False = No = Ghalat = 0<br />
    True = Yes = Dorost = 1
  </p>
\end{frame}
\begin{frame}
  \frametitle{enum}
  \begin{itemize}
    \item After definition of an enumeration
    \item We can use the tag to declare variables
    \item We can use the names to assign values to the variables
  \end{itemize}
  <pre><code class="hljs lang-c">
enum week { sat, sun, mon, tue, wed, thu, fri };
enum week day = sat;
for(day = sat; day &lt;= fri; day++)
  </code></pre>
\end{frame}
\begin{frame}
  \frametitle{Example: Read the number of visitors}
  <pre><code class="hljs lang-c">
enum week {sat, sun, mon, tue, wed, thu, fri};
int visitors[7];
enum week day;
for(day = sat; day &lt;= fri; day++)
  scanf("%d", &amp;visitors[day]);
  </code></pre>
\end{frame}
\begin{frame}
  \frametitle{Caution}
  \begin{itemize}
    \item
      C compiler <span class="hl-orange">does not</span> check the value is
      assigned to the enum variables

    <pre><code class="hljs lang-c">
enum test1 {t1, t2, t3};
enum test2 {t4, t5, t6};
enum test1 t1v = t1;
enum test2 t2v = t4;
if(t1v == t2v) &rarr; true
t1v = t5;
t2v = 100;
    </code></pre>
  \end{itemize}
\end{frame}
\begin{frame}
  <div class="toc" data-selected="5"></div>
\end{frame}
\begin{frame}
  \frametitle{Introduction}
  \begin{itemize}
    \item
      Like Structures,
      <span class="hl-orange">union is a user defined data type</span>.

    \item In union, all members share the same memory location.
  \end{itemize}
\end{frame}
\begin{frame}
  \frametitle{Introduction}
  <img src="img/union.png" />
\end{frame}
\begin{frame}
  <pre><code class="hljs lang-c">
#include &lt;stdio.h&gt;

// Declaration of union is same as structures
union test {
    int x, y;
};

int main()
{
    // A union variable t
    union test t;

    t.x = 2; // t.y also gets value 2
    printf("After making x = 2:\n x = %d, y = %d\n\n",
           t.x, t.y);

    t.y = 10; // t.x is also updated to 10
    printf("After making y = 10:\n x = %d, y = %d\n\n",
           t.x, t.y);
    return 0;
}
  </code></pre>
  <pre><code class="hljs no-highlight">
After making x = 2:
 x = 2, y = 2

After making y = 10:
 x = 10, y = 10
  </code></pre>
\end{frame}
\begin{frame}
  <pre><code class="hljs lang-c">
#include &lt;stdio.h&gt;

union test1 {
    int x;
    int y;
} Test1;

union test2 {
    int x;
    char y;
} Test2;

union test3 {
    int arr[10];
    char y;
} Test3;

int main()
{
    printf("sizeof(test1) = %lu, sizeof(test2) = %lu, "
           "sizeof(test3) = %lu",
           sizeof(Test1),
           sizeof(Test2), sizeof(Test3));
    return 0;
}
  </code></pre>
  <pre><code class="hljs no-highlight">
sizeof(test1) = 4, sizeof(test2) = 4, sizeof(test3) = 40
  </code></pre>
\end{frame}
\begin{frame}
  \begin{frame}
    \frametitle{Legends}
    <img height="300px" src="img/rob.jpg" />
  \end{frame}
  \begin{frame}
    <h6 class="fragment" data-fragment-index="1" style="color: #ff99ff">
      Rob Pike
    </h6>
    <ul class="fragment" data-fragment-index="2">
      \item Canadian programmer and author
      \item
        He is best known for his work on the Go programming language and at Bell
        Labs

    \end{itemize}
  \end{frame}
\end{frame}

\end{document}
