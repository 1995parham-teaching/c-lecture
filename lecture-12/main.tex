\documentclass{../c-lecture}

\subtitle{Files}

\begin{document}

\begin{frame}
  \titlepage{}
\end{frame}
\begin{frame}
  \frametitle{Outline}
  \tableofcontents{}
\end{frame}

\section{Introduction}

\begin{frame}
  \frametitle{Introduction}
  \begin{itemize}
    \item Data storages of computers:
    <ol>
      \item Main memory (RAM)
      \begin{itemize}
        \item It is <span class="hl-orange">volatile</span>
        \item Read / Write data using <span class="hl-green">variables</span>
      \end{itemize}
      \item Secondary storage (Hard Disk)
      \begin{itemize}
        \item
          It is <span class="hl-red">not</span>
          <span class="hl-orange">volatile</span>

        \item Read / Write data using <span class="hl-green">files</span>
      \end{itemize}
    </ol>
  \end{itemize}
\end{frame}
\begin{frame}
  \frametitle{Text &amp; Binary Files}
  \begin{itemize}
    \item How does computer store data?
    \begin{itemize}
      \item They are coded
    \end{itemize}
    \item When data are stored in main memory
    \begin{itemize}
      \item It is variable
      \item Coding is specified by the type: int, char, ...
    \end{itemize}
    \item When data are stored in secondary memory
    \begin{itemize}
      \item It is file
      \item Coding is specified by the file type: Text &amp; Binary
    \end{itemize}
  \end{itemize}
\end{frame}
\begin{frame}
  \frametitle{Text Files}
  \begin{itemize}
    \item ASCII encoding
    \item Each line is a string
    \item Each line is terminated by \n
    \item Human-readable files
    \begin{itemize}
      \item Editable by text editor (e.g. Notepad)
    \end{itemize}
    \item Examples
    \begin{itemize}
      \item C source files
      \item Every .txt files
    \end{itemize}
  \end{itemize}
\end{frame}
\begin{frame}
  \frametitle{Binary Files}
  \begin{itemize}
    \item Binary encoding
    \begin{itemize}
      \item
        int, double, float, struct, ... are directly (as 0,1) stored in the file

    \end{itemize}
    \item Human unreadable files
    \begin{itemize}
      \item Is not editable by text editor
      \item Needs special editor which understands the file
    \end{itemize}
    \item Examples
    \begin{itemize}
      \item .exe files
      \item Media files such as .mp3
      \item Picture files such as .bmp, .jpg
    \end{itemize}
  \end{itemize}
\end{frame}
\begin{frame}
  \frametitle{Working with Files}
  \begin{itemize}
    \item Until now
    \begin{itemize}
      \item We read/write data from/to terminal (console)
    \end{itemize}
    \item In C
    \begin{itemize}
      \item We can read data from file
      \item We can write data to file
    \end{itemize}
  \end{itemize}
\end{frame}
\begin{frame}
  \frametitle{Working with Files}
  <p>Main steps in working with files:</p>
  <ol>
    \item Open file
    \begin{itemize}
      \item Get a file handler from Operating System
    \end{itemize}
    \item Read/Write
    \begin{itemize}
      \item Use the handler
    \end{itemize}
    \item Close file
    \begin{itemize}
      \item Free the handler
    \end{itemize}
    \item Other operations
    \begin{itemize}
      \item Check end of file, skip in file, ...
    \end{itemize}
  </ol>
\end{frame}
\begin{frame}
  \frametitle{Opening Files}
  \begin{itemize}
    \item Function <span class="hl-violet">fopen</span> opens files
    <pre><code class="hljs lang-c">
#include &lt;stdio.h&gt;
FILE* fopen(char* name, char* mode);
    </code></pre>
    \item <span class="hl-orange">FILE*</span> is a struct
    \begin{itemize}
      \item Saves information about file.
      \item We <span class="hl-green">don’t need</span> to know about it.
    \end{itemize}
    \item
      If cannot open file, fopen returns <span class="hl-red">NULL</span>.

    \item name is the name of file:
    \begin{itemize}
      \item <span class="hl-cyan">Absolute</span> name: C:\prog\test.txt
      \item <span class="hl-cyan">Relative</span> name: Mytest.txt
    \end{itemize}
  \end{itemize}
\end{frame}
\begin{frame}
  \begin{frame}
    \frametitle{Opening Files: Modes}
    <table>
      <thead>
        <tr>
          <th>File Mode</th>
          <th>Description</th>
        </tr>
      </thead>
      <tbody>
        <tr>
          <td><span class="hl-cyan">r</span></td>
          <td>
            Open a file for <span class="hl-orange">reading</span>. If a file is
            in reading mode, then
            <span class="hl-green">no data is deleted</span> if a file is
            already present on a system.
          </td>
        </tr>
        <tr>
          <td><span class="hl-cyan">w</span></td>
          <td>
            Open a file for <span class="hl-orange">writing</span>. If a file is
            in writing mode, then a new file is created if a file doesn't exist
            at all. If a file is already present on a system, then
            <span class="hl-green"
              >all the data inside the file is truncated</span
            >, and it is opened for writing purposes.
          </td>
        </tr>
      </tbody>
    </table>
  \end{frame}
  \begin{frame}
    <table>
      <thead>
        <tr>
          <th>File Mode</th>
          <th>Description</th>
        </tr>
      </thead>
      <tbody>
        <tr>
          <td><span class="hl-cyan">a</span></td>
          <td>
            Open a file in append mode. If a file is in append mode, then the
            file is opened. The content within the file doesn't change.
          </td>
        </tr>
        <tr>
          <td><span class="hl-cyan">r+</span></td>
          <td>open for reading and writing from beginning</td>
        </tr>
        <tr>
          <td><span class="hl-cyan">w+</span></td>
          <td>open for reading and writing, overwriting a file</td>
        </tr>
        <tr>
          <td><span class="hl-cyan">a+</span></td>
          <td>open for reading and writing, appending to file</td>
        </tr>
      </tbody>
    </table>
  \end{frame}
\end{frame}
\begin{frame}
  \frametitle{Opening Files: Modes}
  \begin{itemize}
    \item Files are
    \begin{itemize}
      \item Text: Some strings
      \item Binary: Image file, Video file, ...
    \end{itemize}
    \item To open binary file, we should add b to the mode.
    \begin{itemize}
      \item rb : open binary file for read
      \item w+b: create new binary file for read and write
    \end{itemize}
  \end{itemize}
\end{frame}
\begin{frame}
  \frametitle{Opening Files: Example}
  <pre><code class="hljs lang-c">
FILE *fp;
fp = fopen("c:\\test.txt", "r");
if(fp == NULL){
  printf("Cannot open file\n");
  return -1;
}
  </code></pre>
  <p>Open file c:\test.txt for reading</p>
\end{frame}
\begin{frame}
  \frametitle{File-Position Pointer (FPP)}
  \begin{itemize}
    \item File-Position Pointer
    \begin{itemize}
      \item A pointer in file
      \item
        Points to <span class="hl-orange">current location</span> of read and
        write

    \end{itemize}
    \item When file is open
    \begin{itemize}
      \item
        File-Position Pointer is set to
        <span class="hl-cyan">start of file</span> except for
        <span class="hl-green">"a"</span> mode

    \end{itemize}
    \item When you read/write from/to file
    \begin{itemize}
      \item
        The File-Position Pointer
        <span class="hl-violet">advance according to the size of data</span>

      \begin{itemize}
        \item If you read 2 bytes, it moves 2 bytes
        \item If you write 50 bytes, it advances 50 bytes
      \end{itemize}
    \end{itemize}
  \end{itemize}
\end{frame}
\begin{frame}
  \frametitle{Closing Files}
  \begin{itemize}
    \item Each opened file should be closed.
    \item If we write to a file and don’t close it, some of data may be LOST
    \item To close the file
    <pre><code class="hljs lang-c">
fclose(FILE *fp);
    </code></pre>
  \end{itemize}
\end{frame}
\begin{frame}
  <div class="toc" data-selected="1"></div>
\end{frame}
\begin{frame}
  \frametitle{Reading/Writing Text File}
  \begin{itemize}
    \item <span class="hl-orange">fscanf</span> reads from file.
    \item
      <span class="hl-orange">fscanf</span> is same to
      <span class="hl-green">scanf</span>.

    \item Returns EOF if reached to the end of file.
    <hr />
    \item <span class="hl-cyan">fprintf</span> writes to file.
    \item
      <span class="hl-cyan">fprintf</span> is same to
      <span class="hl-violet">printf</span>.

  \end{itemize}
  <pre><code class="hljs lang-c">
int fscanf(FILE *fp,"format", parameters);
int fprintf(FILE *fp,"format", parameters);
  </code></pre>
\end{frame}
\begin{frame}
  \frametitle{Text File: Example}
  <pre><code>
&lt;Number of students&gt;
&lt;id of student 1&gt; &lt;grade of student 1&gt;
&lt;id of student 2&gt; &lt;grade of student 2&gt;
...
&lt;id of student n&gt; &lt;grade of student n&gt;
  </code></pre>
\end{frame}
\begin{frame}
  \frametitle{Average of Students' Grade}
  <pre><code class="hljs lang-c">
#include &lt;stdio.h&gt;
#include &lt;stdlib.h&gt;

int main() {
  FILE* fpin;

  char inname[20];

  int num, i, id;

  double sum, average, grade;

  printf("Enter the name of input file: ");
  scanf("%s", inname);

  fpin = fopen(inname, "r");
  if (fpin == NULL) {
    printf("Cannot open %s\n", inname);
    return -1;
  }

  /* Read the number of students */
  fscanf(fpin, "%d", &amp;num);

  /* Read the id and grade from file */
  sum = 0;
  for(i = 0; i &lt; num; i++){
    fscanf(fpin, "%d %lf", &amp;id, &amp;grade);
    sum += grade;
  }

  average = sum / num;
  printf("Average = %lf\n", average);

  fclose(fpin);
  return 0;
}
  </code></pre>
\end{frame}
\begin{frame}
  \frametitle{Upper Average Students 🤓 v1}
  <pre><code class="hljs lang-c">
#include &lt;stdio.h&gt;
#include &lt;stdlib.h&gt;

int main() {
  FILE* fpin;
  FILE* fpout;

  char inname[20], outname[20];
  int num, i, id;
  double sum, average, grade;

  printf("Enter the name of input file: ");
  scanf("%s", inname);

  printf("Enter the name of output file: ");
  scanf("%s", outname);

  fpin = fopen(inname, "r");
  if (fpin == NULL) {
    printf("Cannot open %s\n", inname);
    return -1;
  }

  fpout = fopen(outname, "w");
  if(fpout == NULL){
    printf("Cannot open %s\n", outname); return -1;
  }

  /* Read the number of students */
  fscanf(fpin, "%d", &amp;num);

  /* Read the id and grade from file */
  sum = 0;
  for(i = 0; i &lt; num; i++){
    fscanf(fpin, "%d %lf", &amp;id, &amp;grade);
    sum += grade;
  }

  average = sum / num;

  fclose(fpin);

  fpin = fopen(inname, "r");

  fscanf(fpin,"%d", &amp;num);
  fprintf(fpout, "%lf\n", average);

  for(i = 0; i &lt; num; i++) {
    fscanf(fpin, "%d %lf", &amp;id, &amp;grade);

    if(grade &gt;= average)
      fprintf(fpout, "%d: %s\n", id, "passed");
    else
      fprintf(fpout, "%d: %s\n", id, "failed");
  }

  fclose(fpin);

  fclose(fpout);

  return 0;
}
  </code></pre>
\end{frame}
\begin{frame}
  \frametitle{Upper Average Students 🤓 v2}
  <pre><code class="hljs lang-c">
#include &lt;tstdio.h&gt;
#include &lt;stdlib.h&gt;

struct student {
  int id;
  double grade;
};

int main() {
  char inname[200];
  char outname[200];
  FILE *fpin;
  FILE *fpout;
  struct student *students;

  int num;
  double sum = 0;

	printf("input file: ");
	scanf("%s", inname);

  printf("output file: ");
  scanf("%s", outname);

	fpin = fopen(inname, "r");
	if (fpin == NULL) {
		printf("cannot open file %s\n", inname);
		return 1;
	}

  fpout = fopen(outname, "w");
  if (fpout == NULL) {
		printf("cannot open file %s\n", outname);
		return 1;
  }

	fscanf(fpin, "%d", &amp;num);

  students = malloc(num * sizeof(struct student));

  for (int i = 0; i &lt; num; i++) {
		int id;
    double grade;

    fscanf(fpin, "%d %lf", &amp;id, &amp;grade);

    sum += grade;

    students[i].id = id;
    students[i].grade = grade;

    printf("id: %d, grade: %lf\n", id, grade);
  }

  double average = sum / num;
  printf("average: %lf\n", average);

  for (int i = 0; i &lt; num; i++) {
    if (students[i].grade &gt;= average) {
      fprintf(fpout, "%d\n", students[i].id);
    }
  }

  fclose(fpin);
}
  </code></pre>
\end{frame}
\begin{frame}
  \frametitle{Reading/Writing Characters (Text Files)}
  \begin{itemize}
    \item To write a character to file
    <pre><code class="hljs lang-c">
fputc(char c, FILE *fp)
    </code></pre>
    \item To read a char from file
    <pre><code class="hljs lang-c">
char fgetc(FILE *fp);
    </code></pre>
  \end{itemize}
  <p>Returns EOF if reaches to End of File</p>
\end{frame}
\begin{frame}
  \frametitle{cp &lt;src&gt; &lt;dst&gt;}
  <pre><code class="hljs lang-c">
#include &lt;stdio.h&gt;
#include &lt;stdlib.h&gt;

int main() {
  FILE* fpin;
  FILE* fpout;
  char inname[20], outname[20];
  char c;

  printf("Enter the name of input file: ");
  scanf("%s", inname);

  printf("Enter the name of output file: ");
  scanf("%s", outname);

  fpin = fopen(inname, "r");
  if (fpin == NULL){
    printf("Cannot open %s\n", inname);
    return -1;
  }

  fpout = fopen(outname, "w");
  if (fpout == NULL) {
    printf("Cannot open %s\n", outname);
    return -1;
  }

  // 😱
  while((c = fgetc(fpin)) != EOF)
    fputc(c, fpout);

  fclose(fpin);
  fclose(fpout);
  return 0;
}
  </code></pre>
\end{frame}
\begin{frame}
  \begin{frame}
    \frametitle{Checking End of File}
    \begin{itemize}
      \item Each file has two indicators
      \begin{itemize}
        \item End of fie indicator
        \item Error indicator
      \end{itemize}
      \item
        These indicators are set when we want to read but there is not enough
        data or there is an error

      \item How to use
      \begin{itemize}
        \item Try to read
        \item If the number of read object is less than expected
        \begin{itemize}
          \item Check end of file &rarr; feof
          \item Check error of file &rarr; ferror
        \end{itemize}
      \end{itemize}
    \end{itemize}
  \end{frame}
  \begin{frame}
    <p>
      feof tells that an attempt has been made to read past the end of the file,
      which is not the same as that we just read the last data item from a file.
      We have to read one past the last data item for feof to return nonzero.
    </p>
  \end{frame}
\end{frame}
\begin{frame}
  \frametitle{Checking End of File}
  <pre><code class="hljs lang-c">
while(1){
  c = fgetc(fpin);
  if(feof(fpin))
    break;
  fputc(c, fpout);
}
  </code></pre>
\end{frame}
\begin{frame}
  \frametitle{Read/Write a Line}
  \begin{itemize}
    \item We can read a line of file
    \begin{itemize}
      \item
        <span class="hl-green">fscanf</span> reads until the first free space

    \end{itemize}
    <pre><code class="hljs lang-c">
char* fgets(char* buff, int maxnumber, FILE* fp);
    </code></pre>
    \item Read at most <span class="hl-orange">maxnumber-1</span> chars
    \item
      Reading stops after <span class="hl-cyan">EOF</span> or
      <span class="hl-cyan">\n</span>

    \item If a <span class="hl-cyan">\n</span> is read it is stored in buffer
    \item Add <span class="hl-violet">\0</span> to the end of string
    \item
      If reach to end of file without reading any character, return
      <span class="hl-red">NULL</span>

  \end{itemize}
\end{frame}
\begin{frame}
  \frametitle{Read/Write a Line}
  \begin{itemize}
    \item We can write a line to file
    <pre><code class="hljs lang-c">
int fputs(char* buff, FILE* fp);
    </code></pre>
    \item Write the string <span class="hl-orange">buff</span> to file
    \item Does <span class="hl-red">NOT</span> add \n at the end
  \end{itemize}
\end{frame}
\begin{frame}
  \frametitle{Read/Write a Line}
  \begin{itemize}
    \item Read a line without knowing its maximum length
    \item
      The caller may provide a pointer to a malloced buffer for the line in
      <span class="hl-orange">*linep</span> , and the capacity of the buffer in
      <span class="hl-green">*linecapp</span>.

    <pre><code class="hljs lang-c">
int getline(char** linep, int* linecapp, FILE* stream);
    </code></pre>
  \end{itemize}
\end{frame}
\begin{frame}
  \frametitle{Example: Count the number of lines}
  <pre><code class="hljs lang-c">
char buf[500]; // 500 &gt; every line length

char inname[] = "1.txt";

FILE* fpin = fopen(inname, "r");
if (fpin == NULL) {
  printf("Cannot open %s\n", inname);
  return -1;
}

int count = 0;

while(fgets(buf, 500, fpin) != NULL)
  count++;

printf("Number of Lines = %d\n", count);
  </code></pre>
\end{frame}
\begin{frame}
  \frametitle{Example: Count the number of lines}
  <pre><code class="hljs lang-c">
char *buf = NULL;

char inname[] = "1.txt";

FILE* fpin = fopen(inname, "r");
if (fpin == NULL) {
  printf("Cannot open %s\n", inname);
  return -1;
}

int count = 0;
int length = 0;

while(getline(&amp;buf, &amp;length, fpin) &gt; -1) {
  count++;
  free(buf);
  buf = NULL;
}

printf("Number of Lines = %d\n", count);
  </code></pre>
\end{frame}
\begin{frame}
  \frametitle{Maximum Line Length Issue}
  <pre><code class="hljs lang-c">
12345678910
123
  </code></pre>
  <p>fgets with maxlen = 10</p>
  <pre><code class="hljs lang-c">
Number of Lines = 3
  </code></pre>
  <p>getline</p>
  <pre><code class="hljs lang-c">
Number of Lines = 2
  </code></pre>
\end{frame}
\begin{frame}
  <div class="toc" data-selected="2"></div>
\end{frame}
\begin{frame}
  \frametitle{Binary Files: A Different File Format}
  \begin{itemize}
    \item Data in binary files are
    \begin{itemize}
      \item Not encoded in ASCII format
      \item Encoded in binary format
    \end{itemize}
    \item We must use different functions to read/write from/to binary files
    \begin{itemize}
      \item Why?
      \item
        Because, data should not be converted to/from ASCII encoding in
        writing/reading the files

    \end{itemize}
  \end{itemize}
\end{frame}
\begin{frame}
  \begin{frame}
    \frametitle{No Conversion to ASCII}
    \begin{itemize}
      \item In text files, everything is saved as ASCII codes
      \begin{itemize}
        <pre><code class="hljs lang-c">
fprintf(fp, “%d”, 10)
        </code></pre>
        \item Saves 2 bytes in the file: ASCII ‘1’ ASCII ‘0’
        <pre><code class="hljs">
00110001 00110000
        </code></pre>
        <pre><code class="hljs lang-c">
fscanf(fp, “%d”, &amp;i)
        </code></pre>
        \item
          Read 2 bytes from file (ASCII ‘1’ ASCII ‘0’) and convert it to base 2
          which mean integer number 10

      \end{itemize}
    \end{itemize}
  \end{frame}
  \begin{frame}
    \frametitle{No Conversion to ASCII}
    \begin{itemize}
      \item
        In binary files, there is not any binary to text conversion, everything
        is read/write in binary format

      \begin{itemize}
        <pre><code class="hljs lang-c">
int i = 10;
fwrite(&amp;i, sizeof(int), 1, fp)
        </code></pre>
        \item Saves 4 bytes in the file: The code of 10 in base 2
        <pre><code class="hljs">
00000000 00000000 00000000 00001010
        </code></pre>
        <pre><code class="hljs lang-c">
fread(&amp;i, sizeof(int), 1, fp)
        </code></pre>
        \item Reads 4 bytes from file into i (without any conversion)
      \end{itemize}
    \end{itemize}
  \end{frame}
\end{frame}
\begin{frame}
  \frametitle{Writing to Binary Files}
  <pre><code class="hljs lang-c">
int fwrite(void *buf, int size, int num, FILE *fp)
  </code></pre>
  \begin{itemize}
    \item
      Writes <span class="hl-orange">num</span> objects from
      <span class="hl-green">buf</span> to <span class="hl-violet">fp</span>.

    \item Size of each object is <span class="hl-yellow">size</span>.
    \item Returns the number of written objects.
    \item If (return_val &lt; <span class="hl-orange">num</span>)
    \begin{itemize}
      \item There is an error
    \end{itemize}
  \end{itemize}
\end{frame}
\begin{frame}
  \frametitle{Reading from Binary Files}
  <pre><code class="hljs lang-c">
int fread(void *buf, int size, int num, FILE *fp)
  </code></pre>
  \begin{itemize}
    \item
      Reads <span class="hl-orange">num</span> objects from file
      <span class="hl-violet">fp</span> to <span class="hl-green">buf</span>.

    \item Size of each object is <span class="hl-yellow">size</span>.
    \item Returns the number of read objects.
    \item If (return val &lt; <span class="hl-orange">num</span>)
    \begin{itemize}
      \item There is an error
      \item Or EOF &rarr; Check with feof
    \end{itemize}
  \end{itemize}
\end{frame}
\begin{frame}
  \frametitle{fread: Examples}
  \begin{itemize}
    \item Reading 1 int from binary file fp
    <pre><code class="hljs lang-c">
int i;
fread(&amp;i, sizeof(int), 1, fp);
    </code></pre>
    \item This means
    \begin{itemize}
      \item
        Read <span class="hl-orange">1</span> object from file
        <span class="hl-violet">fp</span>.

      \item Save result in <span class="hl-green">&amp;i</span>.
      \item
        The size of the object is <span class="hl-yellow">sizeof(int)</span>

    \end{itemize}
    \item It reads 4 bytes from file and saves in &amp;i
    \begin{itemize}
      \item We read an integer from file and save it in i
    \end{itemize}
  \end{itemize}
\end{frame}
\begin{frame}
  \frametitle{fread: Examples}
  \begin{itemize}
    \item Read five floats
    <pre><code class="hljs lang-c">
float farr[5];
fread(farr, sizeof(float), 5, fp);
    </code></pre>
    \item This means
    \begin{itemize}
      \item
        Read <span class="hl-orange">5</span> objects from file
        <span class="hl-violet">fp</span>.

      \item Save result in <span class="hl-green">farr</span>.
      \item
        The size of each object is
        <span class="hl-yellow">sizeof(float)</span>

    \end{itemize}
    \item It reads 20 bytes from file and saves in farr
    \begin{itemize}
      \item We read 5 floats from file and save them in farr
    \end{itemize}
  \end{itemize}
\end{frame}
\begin{frame}
  \frametitle{fwrite: Examples}
  \begin{itemize}
    \item Writing 1 char to binary file fp
    <pre><code class="hljs lang-c">
char c = 'A';
fwrite(&amp;c, sizeof(char), 1, fp);
    </code></pre>
    \item This means
    \begin{itemize}
      \item
        Write <span class="hl-orange">1</span> object from
        <span class="hl-green">&amp;c</span> into file
        <span class="hl-violet">fp</span>.

      \item Size of the object is <span class="hl-yellow">sizeof(char)</span>
    \end{itemize}
    \item It writes 1 byte from address &amp;c and saves result in file
    \begin{itemize}
      \item We write char c to the file
    \end{itemize}
  \end{itemize}
\end{frame}
\begin{frame}
  \frametitle{fwrite: Examples}
  \begin{itemize}
    \item Writing 4 doubles to binary file fp
    <pre><code class="hljs lang-c">
double darr[4];
fwrite(darr, sizeof(double), 4, fp);
    </code></pre>
    \item This means
    \begin{itemize}
      \item
        Write <span class="hl-orange">4</span> object from
        <span class="hl-green">darr</span> into file
        <span class="hl-violet">fp</span>.

      \item
        Size of the object is <span class="hl-yellow">sizeof(double)</span>

    \end{itemize}
    \item It writes 32 bytes from address darr and saves result in file
    \begin{itemize}
      \item We write the array of double to the file
    \end{itemize}
  \end{itemize}
\end{frame}
\begin{frame}
  \frametitle{}
  <pre><code class="hljs lang-c">
#include &lt;stdio.h&gt;

struct point{
  int x, y;
};

int main() {
  FILE* fp;
  struct point p;
  int i;

  fp = fopen("points.bin", "wb");
  if (fp == NULL) {
    printf("Cannot create file\n");
    return -1;
  }

  for(i = 0; i &lt; 5; i++) {
    printf("Enter X and Y: ");
    scanf("%d %d", &amp;p.x, &amp;p.y);
    fwrite(&amp;p, sizeof(p), 1, fp);
  }

  fclose(fp);
  return 0;
}
  </code></pre>
\end{frame}
\begin{frame}
  \frametitle{}
  <pre><code class="hljs lang-c">
#include &lt;stdio.h&gt;
struct point{
  int x, y;
};

int main() {
  FILE* fp;
  struct point p;
  int i;
  fp = fopen("points.bin", "rb");
  if(fp == NULL){
    printf("Cannot read from file\n");
    return -1;
  }

  while(1){
    if(fread(&amp;p, sizeof(p), 1, fp) &lt; 1)
      break;
    printf("X = %d, and Y = %d\n", p.x, p.y);
  }

  fclose(fp);
  return 0;
}
  </code></pre>
\end{frame}
\begin{frame}
  \frametitle{Sequential and Random Accesses}
  \begin{itemize}
    \item The access to file is sequential if
    \begin{itemize}
      \item If we don’t move the FPP manually
      \item FPP advances through read and write
    \end{itemize}
    \item The access to file is Random
    \begin{itemize}
      \item FPP advances through read and write
      \item We can also move the FPP manually
    \end{itemize}
    \item File processing can uses Random access
  \end{itemize}
\end{frame}
\begin{frame}
  \frametitle{Moving FPP, Why?}
  \begin{itemize}
    \item To access randomly
    \item
      Consider very large file (information about all students in the
      university)

    \item Change the name of 5000th student
    \begin{itemize}
      \item If it is saved in text file
      \begin{itemize}
        \item Read 4999 lines, skip them and change the 5000th
      \end{itemize}
      \item If it is saved in binary file and each object has the same size
      \begin{itemize}
        \item Jump to the 5000th object by fseek
        \item Why we cannot do the same for text files? More on this later
      \end{itemize}
    \end{itemize}
  \end{itemize}
\end{frame}
\begin{frame}
  \frametitle{Moving FPP}
  <pre><code class="hljs lang-c">
int fseek(FILE *fp, long offset, int org)
  </code></pre>
  \begin{itemize}
    \item Set FPP in the offset respect to org
    \item org:
    \begin{itemize}
      \item SEEK_SET: start of file
      \item SEEK_CUR: current FPP
      \item SEEK_END: End of file
    \end{itemize}
    \item Returns nonzero if it is unsuccessful
  \end{itemize}
\end{frame}
\begin{frame}
  <pre><code class="hljs">
(1,1)(2,2)(3,3)(4,4)(5,5)
  </code></pre>
  <pre><code class="hljs lang-c">
fp = fopen("points.bin", "rb");

fread(&amp;p, sizeof(p), 1, fp);
printf("%d %d\n", p.x, p.y); // 1 1

fseek(fp, 2 * sizeof(p), SEEK_SET);
fread(&amp;p, sizeof(p), 1, fp);
printf("%d %d\n", p.x, p.y); // 3 3

fseek(fp, -3 * sizeof(p), SEEK_END);
fread(&amp;p, sizeof(p), 1,fp);
printf("%d %d\n", p.x, p.y); // 3 3

fseek(fp, 1 * sizeof(p), SEEK_CUR);
fread(&amp;p, sizeof(p), 1, fp); // 5 5
printf("%d %d\n", p.x, p.y);
  </code></pre>
\end{frame}
\begin{frame}
  \frametitle{Other FPP related functions}
  \begin{itemize}
    \item Find out where is the FPP
    <pre><code class="hljs lang-c">
int ftell(FILE *fp)
    </code></pre>
    \item ftell returns the current FPP
    \begin{itemize}
      \item With respect to SEEK_SET
    \end{itemize}
    \item Reset the FPP to the start of file
    <pre><code class="hljs lang-c">
void rewind(FILE *fp)
    </code></pre>
  \end{itemize}
\end{frame}
\begin{frame}
  <pre><code class="hljs lang-c">
#include &lt;stdio.h&gt;

struct point{
  int x, y;
};

int main(void){
   FILE* fp;

  struct point p;
  int num;

  fp = fopen("points.bin", "rb+");
  if (fp == NULL){
    printf("Cannot read from file\n");
    return -1;
  }

  printf("Enter the number of points: ");
  scanf("%d", &amp;num);

  printf("Enter new X and Y: ");
  scanf("%d %d", &amp;(p.x), &amp;(p.y));

  fseek(fp, (num – 1) * sizeof(p), SEEK_SET);
  fwrite(&amp;p, sizeof(p), 1, fp);

  fclose(fp);

  return 0;
}
  </code></pre>
\end{frame}
\begin{frame}
  \frametitle{fseek in Text files}
  \begin{itemize}
    \item Not very useful
    \item Offset counts the number of characters including ‘\n’
    \item Typical useful versions
    \begin{itemize}
      \item Go to the start of file
      <pre><code class="hljs lang-c">
fseek(fp, 0, SEEK_SET)
      </code></pre>
      \item Go to the end of file
      <pre><code class="hljs lang-c">
fseek(fp, 0, SEEK_END)
      </code></pre>
    \end{itemize}
  \end{itemize}
\end{frame}
\begin{frame}
  <div class="toc" data-selected="3"></div>
\end{frame}
\begin{frame}
  \frametitle{Common Bugs and Avoiding Them}
  \begin{itemize}
    \item Take care about mode in fopen
    \begin{itemize}
      \item w &amp; w+: all data in file will be lost
      \item r: you cannot write. fprintf does not do any thing
    \end{itemize}
    \item Take care about text or binary
    \begin{itemize}
      \item fscanf/fprintf don’t do meaningful job in binary files
    \end{itemize}
    \item Check the successful open: fp != NULL
    \item Check EOF as much as possible.
    \item Close the open files.
  \end{itemize}
\end{frame}

\begin{frame}
  \frametitle{Legends}
  <img src="./img/yoda.jpg" alt="yoda" />
\end{frame}
\begin{frame}
  <h6 class="fragment" data-fragment-index="1" style="color: #ff99ff">You</h6>
  \begin{itemize}
    \item Thanks for attending this course
  \end{itemize}
\end{frame}

\end{document}
