\documentclass{../c-lecture}

\subtitle{Files}

\usepackage{tabularx}

\begin{document}

\begin{frame}
  \titlepage{}
\end{frame}
\begin{frame}
  \frametitle{Outline}
  \tableofcontents{}
\end{frame}

\section{Introduction}

\begin{frame}
  \frametitle{Introduction}
  \begin{itemize}
    \item Data storages of computers:
    \begin{enumerate}
      \item Main memory (RAM)
      \begin{itemize}
        \item It is \textit{\color{YellowOrange} volatile}
        \item Read / Write data using \textit{\color{LimeGreen} variables}
      \end{itemize}
      \item Secondary storage (Hard Disk)
      \begin{itemize}
        \item
          It is \textsc{\color{RubineRed} not}
          \textit{\color{YellowOrange} volatile}

        \item Read / Write data using \textit{\color{LimeGreen} files}
      \end{itemize}
    \end{enumerate}
  \end{itemize}
\end{frame}

\begin{frame}
  \frametitle{Text \& Binary Files}
  \begin{itemize}
    \item How does computer store data?
    \begin{itemize}
      \item They are coded
    \end{itemize}
    \item When data are stored in main memory
    \begin{itemize}
      \item It is variable
      \item Coding is specified by the type: int, char, \ldots
    \end{itemize}
    \item When data are stored in secondary memory
    \begin{itemize}
      \item It is file
      \item Coding is specified by the file type: Text \& Binary
    \end{itemize}
  \end{itemize}
\end{frame}

\begin{frame}[fragile]
  \frametitle{Text Files}
  \begin{itemize}
    \item ASCII encoding
    \item Each line is a string
    \item Each line is terminated by \mintinline{c}|'\n'|
    \item Human-readable files
    \begin{itemize}
      \item Editable by text editor (e.g. Notepad)
    \end{itemize}
    \item Examples
    \begin{itemize}
      \item C source files
      \item Every .txt files
    \end{itemize}
  \end{itemize}
\end{frame}

\begin{frame}
  \frametitle{Binary Files}
  \begin{itemize}
    \item Binary encoding
    \begin{itemize}
      \item
        int, double, float, struct, \ldots are directly (as 0,1) stored in the file
    \end{itemize}
    \item Human \textcolor{RubineRed}{un}readable files
    \begin{itemize}
      \item Is not editable by text editor
      \item Needs special editor which understands the file
    \end{itemize}
    \item Examples
    \begin{itemize}
      \item .exe files
      \item Media files such as .mp3
      \item Picture files such as .bmp, .jpg
    \end{itemize}
  \end{itemize}
\end{frame}

\begin{frame}
  \frametitle{Working with Files}
  \begin{itemize}
    \item Until now
    \begin{itemize}
      \item We read/write data from/to terminal (console)
    \end{itemize}
    \item In C
    \begin{itemize}
      \item We can read data from file
      \item We can write data to file
    \end{itemize}
  \end{itemize}
\end{frame}

\begin{frame}
  \frametitle{Working with Files}
  Main steps in working with files:
  \begin{enumerate}
    \item Open file
    \begin{itemize}
      \item Get a file handler from Operating System
    \end{itemize}
    \item Read/Write
    \begin{itemize}
      \item Use the handler
    \end{itemize}
    \item Close file
    \begin{itemize}
      \item Free the handler
    \end{itemize}
    \item Other operations
    \begin{itemize}
      \item Check end of file, skip in file, \ldots
    \end{itemize}
  \end{enumerate}
\end{frame}

\begin{frame}[fragile]
  \frametitle{Opening Files}
  \begin{itemize}
    \item Function \textit{\color{Violet} fopen} opens files
    \begin{minted}[bgcolor=Black]{c}
#include <stdio.h>
FILE* fopen(char* name, char* mode);
    \end{minted}
    \item \textit{\color{YellowOrange} FILE*} is a struct
    \begin{itemize}
      \item Saves information about file.
      \item We \textsc{\color{RubineRed} don’t need} to know about it.
    \end{itemize}
    \item
      If cannot open file, fopen returns \textit{\color{YellowOrange} NULL}.
    \item name is the name of file:
    \begin{itemize}
      \item \textit{\color{Cyan} Absolute} name: C:\textbackslash prog\textbackslash test.txt
      \item \textit{\color{Cyan} Relative} name: Mytest.txt
    \end{itemize}
  \end{itemize}
\end{frame}

\begin{frame}
  \frametitle{Opening Files: Modes}
  \begin{table}
  \begin{tabularx}{\textwidth}{cX}
    \toprule
    File Mode &
    Description \\
    \midrule
    \textcolor{Cyan}{r} &
    Open a file for \textbf{\color{YellowOrange} reading}. If a file is
    in reading mode, then
    \textbf{\color{LimeGreen} no data is deleted} if a file is
    already present on a system.\\
    \midrule
    \textcolor{Cyan}{w} &
    Open a file for \textbf{\color{YellowOrange} writing}. If a file is
    in writing mode, then a new file is created if a file doesn't exist
    at all. If a file is already present on a system, then
    \textbf{\color{LimeGreen} all the data inside the file is truncated},
    and it is opened for writing purposes.\\
    \bottomrule
  \end{tabularx}
  \end{table}
\end{frame}

\begin{frame}
  \begin{table}
  \begin{tabularx}{\textwidth}{cX}
    \toprule
    File Mode &
    Description \\
    \midrule
    \textcolor{Cyan}{a} &
    Open a file in append mode. If a file is in append mode, then the
    file is opened. The content within the file doesn't change. \\
    \midrule
    \textcolor{Cyan}{r+} &
    open for reading and writing from beginning \\
    \midrule
    \textcolor{Cyan}{w+} &
    open for reading and writing, overwriting a file \\
    \midrule
    \textcolor{Cyan}{a+} &
    open for reading and writing, appending to file \\
    \bottomrule
  \end{tabularx}
  \end{table}
\end{frame}

\begin{frame}
  \frametitle{Opening Files: Modes}
  \begin{itemize}
    \item Files are
    \begin{itemize}
      \item Text: Some strings
      \item Binary: Image file, Video file, \ldots
    \end{itemize}
    \item To open binary file, we should add b to the mode.
    \begin{itemize}
      \item rb : open binary file for read
      \item w+b: create new binary file for read and write
    \end{itemize}
  \end{itemize}
\end{frame}

\begin{frame}[fragile]
  \frametitle{Opening Files: Example}
  \begin{minted}[bgcolor=Black]{c}
FILE *fp;
fp = fopen("c:\\test.txt", "r");
if(fp == NULL){
  printf("Cannot open file\n");
  return -1;
}
  \end{minted}
  Open file c:\textbackslash test.txt for reading
\end{frame}

\begin{frame}
  \frametitle{File-Position Pointer (FPP)}
  \begin{itemize}
    \item File-Position Pointer
    \begin{itemize}
      \item A pointer in file
      \item
        Points to \textit{\color{YellowOrange} current location} of read and
        write

    \end{itemize}
    \item When file is open
    \begin{itemize}
      \item
        File-Position Pointer is set to
        \textit{\color{Cyan} start of file} except for
        \textit{\color{LimeGreen}"a"} mode

    \end{itemize}
    \item When you read/write from/to file
    \begin{itemize}
      \item
        The File-Position Pointer
        \textit{\color{Violet} advance according to the size of data}.

      \begin{itemize}
        \item If you read 2 bytes, it moves 2 bytes
        \item If you write 50 bytes, it advances 50 bytes
      \end{itemize}
    \end{itemize}
  \end{itemize}
\end{frame}

\begin{frame}[fragile]
  \frametitle{Closing Files}
  \begin{itemize}
    \item Each opened file should be closed.
    \item If we write to a file and don’t close it, some of data may be LOST
    \item To close the file
    \begin{minted}[bgcolor=Black]{c}
fclose(FILE *fp);
    \end{minted}
  \end{itemize}
\end{frame}

\section{Text File Operations}

\begin{frame}[fragile]
  \frametitle{Reading/Writing Text File}
  \begin{itemize}
    \item \textit{\color{YellowOrange} fscanf} reads from file.
    \item
      \textit{\color{YellowOrange}fscanf} is same to
      \textit{\color{LimeGreen} scanf}.

    \item Returns EOF if reached to the end of file.
    \item \textit{\color{Cyan} fprintf} writes to file.
    \item
      \textit{\color{Cyan} fprintf} is same to
      \textit{\color{Violet} printf}.

  \end{itemize}
  \begin{minted}[bgcolor=Black]{c}
int fscanf(FILE *fp,"format", parameters);
int fprintf(FILE *fp,"format", parameters);
  \end{minted}
\end{frame}

\begin{frame}[fragile]
  \frametitle{Text File: Example}
  \begin{minted}[bgcolor=Black]{c}
<Number of students>
<id of student 1> <grade of student 1>
<id of student 2> <grade of student 2>
...
<id of student n> <grade of student n>
  \end{minted}
\end{frame}

\begin{frame}[fragile]
  \frametitle{Average of Students' Grade}
  \scriptsize
  \begin{minted}[bgcolor=Black]{c}
#include <stdio.h>
#include <stdlib.h>

int main() {
  FILE* fpin;

  char inname[20];

  int num, i, id;

  double sum, average, grade;

  printf("Enter the name of input file: ");
  scanf("%s", inname);

  fpin = fopen(inname, "r");
  if (fpin == NULL) {
    printf("Cannot open %s\n", inname);
    return -1;
  }

  /* Read the number of students */
  fscanf(fpin, "%d", &num);

  /* Read the id and grade from file */
  sum = 0;
  for(i = 0; i < num; i++){
    fscanf(fpin, "%d %lf", &id, &grade);
    sum += grade;
  }

  average = sum / num;
  printf("Average = %lf\n", average);

  fclose(fpin);
  return 0;
}
  \end{minted}
\end{frame}

\begin{frame}[fragile]
  \frametitle{Upper Average Students 🤓 v1}
  \scriptsize
  \begin{minted}[bgcolor=Black]{c}
#include <stdio.h>
#include <stdlib.h>

int main() {
  FILE* fpin;
  FILE* fpout;

  char inname[20], outname[20];
  int num, i, id;
  double sum, average, grade;

  printf("Enter the name of input file: ");
  scanf("%s", inname);

  printf("Enter the name of output file: ");
  scanf("%s", outname);

  fpin = fopen(inname, "r");
  if (fpin == NULL) {
    printf("Cannot open %s\n", inname);
    return -1;
  }

  fpout = fopen(outname, "w");
  if(fpout == NULL){
    printf("Cannot open %s\n", outname); return -1;
  }

  /* Read the number of students */
  fscanf(fpin, "%d", &num);

  /* Read the id and grade from file */
  sum = 0;
  for(i = 0; i < num; i++){
    fscanf(fpin, "%d %lf", &id, &grade);
    sum += grade;
  }

  average = sum / num;

  fclose(fpin);

  fpin = fopen(inname, "r");

  fscanf(fpin,"%d", &num);
  fprintf(fpout, "%lf\n", average);

  for(i = 0; i < num; i++) {
    fscanf(fpin, "%d %lf", &id, &grade);

    if(grade >= average)
      fprintf(fpout, "%d: %s\n", id, "passed");
    else
      fprintf(fpout, "%d: %s\n", id, "failed");
  }

  fclose(fpin);

  fclose(fpout);

  return 0;
}
  \end{minted}
\end{frame}

\begin{frame}[fragile]
  \scriptsize
  \frametitle{Upper Average Students 🤓 v2}
  \begin{minted}[bgcolor=Black]{c}
#include <tstdio.h>
#include <stdlib.h>

struct student {
  int id;
  double grade;
};

int main() {
  char inname[200];
  char outname[200];
  FILE *fpin;
  FILE *fpout;
  struct student *students;

  int num;
  double sum = 0;

	printf("input file: ");
	scanf("%s", inname);

  printf("output file: ");
  scanf("%s", outname);

	fpin = fopen(inname, "r");
	if (fpin == NULL) {
		printf("cannot open file %s\n", inname);
		return 1;
	}

  fpout = fopen(outname, "w");
  if (fpout == NULL) {
		printf("cannot open file %s\n", outname);
		return 1;
  }

	fscanf(fpin, "%d", &num);

  students = malloc(num * sizeof(struct student));

  for (int i = 0; i < num; i++) {
		int id;
    double grade;

    fscanf(fpin, "%d %lf", &id, &grade);

    sum += grade;

    students[i].id = id;
    students[i].grade = grade;

    printf("id: %d, grade: %lf\n", id, grade);
  }

  double average = sum / num;
  printf("average: %lf\n", average);

  for (int i = 0; i < num; i++) {
    if (students[i].grade >= average) {
      fprintf(fpout, "%d\n", students[i].id);
    }
  }

  fclose(fpin);
}
  \end{minted}
\end{frame}

\begin{frame}[fragile]
  \frametitle{Reading/Writing Characters (Text Files)}
  \begin{itemize}
    \item To write a character to file
    \begin{minted}[bgcolor=Black]{c}
fputc(char c, FILE *fp)
    \end{minted}
    \item To read a char from file
    \begin{minted}[bgcolor=Black]{c}
char fgetc(FILE *fp);
    \end{minted}
  \end{itemize}
  <p>Returns EOF if reaches to End of File</p>
\end{frame}

\begin{frame}[fragile]
  \frametitle{cp <src> <dst>}
  \scriptsize
  \begin{minted}[bgcolor=Black]{c}
#include <stdio.h>
#include <stdlib.h>

int main() {
  FILE* fpin;
  FILE* fpout;
  char inname[20], outname[20];
  char c;

  printf("Enter the name of input file: ");
  scanf("%s", inname);

  printf("Enter the name of output file: ");
  scanf("%s", outname);

  fpin = fopen(inname, "r");
  if (fpin == NULL){
    printf("Cannot open %s\n", inname);
    return -1;
  }

  fpout = fopen(outname, "w");
  if (fpout == NULL) {
    printf("Cannot open %s\n", outname);
    return -1;
  }

  // 😱
  while((c = fgetc(fpin)) != EOF)
    fputc(c, fpout);

  fclose(fpin);
  fclose(fpout);
  return 0;
}
  \end{minted}
\end{frame}

\begin{frame}
  \frametitle{Checking End of File}
  \begin{itemize}
    \item Each file has two indicators
    \begin{itemize}
      \item End of fie indicator
      \item Error indicator
    \end{itemize}
    \item
      These indicators are set when we want to read but there is not enough
      data or there is an error
    \item How to use
    \begin{itemize}
      \item Try to read
      \item If the number of read object is less than expected
      \begin{itemize}
        \item Check end of file \textrightarrow \textit{\color{YellowOrange} feof}
        \item Check error of file \textrightarrow \textit{\color{LimeGreen} ferror}
      \end{itemize}
    \end{itemize}
  \end{itemize}
\end{frame}

\begin{frame}
  \begin{block}
    feof tells that an attempt has been made to read past the end of the file,
    which is not the same as that we just read the last data item from a file.
    We have to read one past the last data item for feof to return nonzero.
  \end{block}
\end{frame}

\begin{frame}[fragile]
  \frametitle{Checking End of File}
  \begin{minted}[bgcolor=Black]{c}
while(1){
  c = fgetc(fpin);
  if(feof(fpin))
    break;
  fputc(c, fpout);
}
  \end{minted}
\end{frame}

\begin{frame}[fragile]
  \frametitle{Read/Write a Line}
  \begin{itemize}
    \item We can read a line of file
    \begin{itemize}
      \item
        \textit{\color{LimeGreen} fscanf} reads until the first free space

    \end{itemize}
    \begin{minted}[bgcolor=Black]{c}
char* fgets(char* buff, int maxnumber, FILE* fp);
    \end{minted}
    \item Read at most \textit{\color{YellowOrange} maxnumber-1} chars
    \item
      Reading stops after \textit{\color{Cyan} EOF} or
      \mintinline{c}|'\n'|

    \item If a \mintinline{c}|'\n'| is read it is stored in buffer
    \item Add \mintinline{c}|'\0'| to the end of string
    \item
      If reach to end of file without reading any character, return
      \textit{\color{Cyan} NULL}
  \end{itemize}
\end{frame}

\begin{frame}[fragile]
  \frametitle{Read/Write a Line}
  \begin{itemize}
    \item We can write a line to file
    \begin{minted}[bgcolor=Black]{c}
int fputs(char* buff, FILE* fp);
    \end{minted}
    \item Write the string \textit{\color{YellowOrange} buff} to file
    \item Does \textsc{\color{RubineRed} NOT} add \mintinline{c}|'\n'| at the end
  \end{itemize}
\end{frame}

\begin{frame}[fragile]
  \frametitle{Read/Write a Line}
  \begin{itemize}
    \item Read a line without knowing its maximum length
    \item
      The caller may provide a pointer to a malloced buffer for the line in
      \textit{\color{YellowOrange}*linep}, and the capacity of the buffer in
      \textit{\color{LimeGreen}*linecapp}.

    \begin{minted}[bgcolor=Black]{c}
int getline(char** linep, int* linecapp, FILE* stream);
    \end{minted}
  \end{itemize}
\end{frame}

\begin{frame}[fragile]
  \frametitle{Example: Count the number of lines}
  \scriptsize
  \begin{minted}[bgcolor=Black]{c}
char buf[500]; // 500 > every line length

char inname[] = "1.txt";

FILE* fpin = fopen(inname, "r");
if (fpin == NULL) {
  printf("Cannot open %s\n", inname);
  return -1;
}

int count = 0;

while(fgets(buf, 500, fpin) != NULL)
  count++;

printf("Number of Lines = %d\n", count);
  \end{minted}
\end{frame}

\begin{frame}[fragile]
  \frametitle{Example: Count the number of lines}
  \scriptsize
  \begin{minted}[bgcolor=Black]{c}
char *buf = NULL;

char inname[] = "1.txt";

FILE* fpin = fopen(inname, "r");
if (fpin == NULL) {
  printf("Cannot open %s\n", inname);
  return -1;
}

int count = 0;
int length = 0;

while(getline(&buf, &length, fpin) > -1) {
  count++;
  free(buf);
  buf = NULL;
}

printf("Number of Lines = %d\n", count);
  \end{minted}
\end{frame}

\begin{frame}[fragile]
  \frametitle{Maximum Line Length Issue}
  \begin{minted}[bgcolor=Black]{c}
12345678910
123
  \end{minted}
  fgets with maxlen = 10
  \begin{minted}[bgcolor=Black]{c}
Number of Lines = 3
  \end{minted}
  getline
  \begin{minted}[bgcolor=Black]{c}
Number of Lines = 2
  \end{minted}
\end{frame}

\section{Binary File Operations}

\begin{frame}
  \frametitle{Binary Files: A Different File Format}
  \begin{itemize}
    \item Data in binary files are
    \begin{itemize}
      \item Not encoded in ASCII format
      \item Encoded in binary format
    \end{itemize}
    \item We must use different functions to read/write from/to binary files
    \begin{itemize}
      \item Why?
      \item
        Because, data should not be converted to/from ASCII encoding in
        writing/reading the files
    \end{itemize}
  \end{itemize}
\end{frame}

\begin{frame}[fragile]
  \frametitle{No Conversion to ASCII}
  \begin{itemize}
    \item In text files, everything is saved as ASCII codes
    \begin{itemize}
      \begin{minted}[bgcolor=Black]{c}
fprintf(fp, "%d", 10)
      \end{minted}
      \item Saves 2 bytes in the file: ASCII `1` ASCII `0`
      \begin{minted}[bgcolor=Black]{output}
00110001 00110000
      \end{minted}
      \begin{minted}[bgcolor=Black]{c}
fscanf(fp, "%d", &i)
      \end{minted}
      \item
        Read 2 bytes from file (ASCII `1` ASCII `0`) and convert it to base 2
        which mean integer number 10
    \end{itemize}
  \end{itemize}
\end{frame}

\begin{frame}[fragile]
  \frametitle{No Conversion to ASCII}
  \begin{itemize}
    \item
      In binary files, there is not any binary to text conversion, everything
      is read/write in binary format

    \begin{itemize}
      \begin{minted}[bgcolor=Black]{c}
int i = 10;
fwrite(&i, sizeof(int), 1, fp)
      \end{minted}
      \item Saves 4 bytes in the file: The code of 10 in base 2
      \begin{minted}[bgcolor=Black]{output}
00000000 00000000 00000000 00001010
      \end{minted}
      \begin{minted}[bgcolor=Black]{c}
fread(&i, sizeof(int), 1, fp)
      \end{minted}
      \item Reads 4 bytes from file into i (without any conversion)
    \end{itemize}
  \end{itemize}
\end{frame}

\begin{frame}[fragile]
  \frametitle{Writing to Binary Files}
  \begin{minted}[bgcolor=Black]{c}
int fwrite(void *buf, int size, int num, FILE *fp)
  \end{minted}
  \begin{itemize}
    \item
      Writes <span class="hl-orange">num</span> objects from
      <span class="hl-green">buf</span> to <span class="hl-violet">fp</span>.

    \item Size of each object is <span class="hl-yellow">size</span>.
    \item Returns the number of written objects.
    \item If (return\_val < <span class="hl-orange">num</span>)
    \begin{itemize}
      \item There is an error
    \end{itemize}
  \end{itemize}
\end{frame}

\begin{frame}[fragile]
  \frametitle{Reading from Binary Files}
  \begin{minted}[bgcolor=Black]{c}
int fread(void *buf, int size, int num, FILE *fp)
  \end{minted}
  \begin{itemize}
    \item
      Reads <span class="hl-orange">num</span> objects from file
      <span class="hl-violet">fp</span> to <span class="hl-green">buf</span>.

    \item Size of each object is <span class="hl-yellow">size</span>.
    \item Returns the number of read objects.
    \item If (return val < <span class="hl-orange">num</span>)
    \begin{itemize}
      \item There is an error
      \item Or EOF \textrightarrow Check with feof
    \end{itemize}
  \end{itemize}
\end{frame}

\begin{frame}[fragile]
  \frametitle{fread: Examples}
  \begin{itemize}
    \item Reading 1 int from binary file fp
    \begin{minted}[bgcolor=Black]{c}
int i;
fread(&i, sizeof(int), 1, fp);
    \end{minted}
    \item This means
    \begin{itemize}
      \item
        Read <span class="hl-orange">1</span> object from file
        <span class="hl-violet">fp</span>.

      \item Save result in \mintinline{c}|&i|.
      \item
        The size of the object is <span class="hl-yellow">sizeof(int)</span>

    \end{itemize}
    \item It reads 4 bytes from file and saves in \mintinline{c}|&i|
    \begin{itemize}
      \item We read an integer from file and save it in i
    \end{itemize}
  \end{itemize}
\end{frame}

\begin{frame}[fragile]
  \frametitle{fread: Examples}
  \begin{itemize}
    \item Read five floats
    \begin{minted}[bgcolor=Black]{c}
float farr[5];
fread(farr, sizeof(float), 5, fp);
    \end{minted}
    \item This means
    \begin{itemize}
      \item
        Read <span class="hl-orange">5</span> objects from file
        <span class="hl-violet">fp</span>.

      \item Save result in <span class="hl-green">farr</span>.
      \item
        The size of each object is
        <span class="hl-yellow">sizeof(float)</span>

    \end{itemize}
    \item It reads 20 bytes from file and saves in farr
    \begin{itemize}
      \item We read 5 floats from file and save them in farr
    \end{itemize}
  \end{itemize}
\end{frame}

\begin{frame}[fragile]
  \frametitle{fwrite: Examples}
  \begin{itemize}
    \item Writing 1 char to binary file fp
    \begin{minted}[bgcolor=Black]{c}
char c = 'A';
fwrite(&c, sizeof(char), 1, fp);
    \end{minted}
    \item This means
    \begin{itemize}
      \item
        Write <span class="hl-orange">1</span> object from
        <span class="hl-green">\&c</span> into file
        <span class="hl-violet">fp</span>.

      \item Size of the object is <span class="hl-yellow">sizeof(char)</span>
    \end{itemize}
    \item It writes 1 byte from address \&c and saves result in file
    \begin{itemize}
      \item We write char c to the file
    \end{itemize}
  \end{itemize}
\end{frame}

\begin{frame}[fragile]
  \frametitle{fwrite: Examples}
  \begin{itemize}
    \item Writing 4 doubles to binary file fp
    \begin{minted}[bgcolor=Black]{c}
double darr[4];
fwrite(darr, sizeof(double), 4, fp);
    \end{minted}
    \item This means
    \begin{itemize}
      \item
        Write <span class="hl-orange">4</span> object from
        <span class="hl-green">darr</span> into file
        <span class="hl-violet">fp</span>.

      \item
        Size of the object is <span class="hl-yellow">sizeof(double)</span>

    \end{itemize}
    \item It writes 32 bytes from address darr and saves result in file
    \begin{itemize}
      \item We write the array of double to the file
    \end{itemize}
  \end{itemize}
\end{frame}

\begin{frame}[fragile]
  \frametitle{}
  \begin{minted}[bgcolor=Black]{c}
#include <stdio.h>

struct point{
  int x, y;
};

int main() {
  FILE* fp;
  struct point p;
  int i;

  fp = fopen("points.bin", "wb");
  if (fp == NULL) {
    printf("Cannot create file\n");
    return -1;
  }

  for(i = 0; i < 5; i++) {
    printf("Enter X and Y: ");
    scanf("%d %d", &p.x, &p.y);
    fwrite(&p, sizeof(p), 1, fp);
  }

  fclose(fp);
  return 0;
}
  \end{minted}
\end{frame}

\begin{frame}[fragile]
  \frametitle{}
  \begin{minted}[bgcolor=Black]{c}
#include <stdio.h>
struct point{
  int x, y;
};

int main() {
  FILE* fp;
  struct point p;
  int i;
  fp = fopen("points.bin", "rb");
  if(fp == NULL){
    printf("Cannot read from file\n");
    return -1;
  }

  while(1){
    if(fread(&p, sizeof(p), 1, fp) < 1)
      break;
    printf("X = %d, and Y = %d\n", p.x, p.y);
  }

  fclose(fp);
  return 0;
}
  \end{minted}
\end{frame}

\begin{frame}
  \frametitle{Sequential and Random Accesses}
  \begin{itemize}
    \item The access to file is sequential if
    \begin{itemize}
      \item If we don’t move the FPP manually
      \item FPP advances through read and write
    \end{itemize}
    \item The access to file is Random
    \begin{itemize}
      \item FPP advances through read and write
      \item We can also move the FPP manually
    \end{itemize}
    \item File processing can uses Random access
  \end{itemize}
\end{frame}

\begin{frame}
  \frametitle{Moving FPP, Why?}
  \begin{itemize}
    \item To access randomly
    \item
      Consider very large file (information about all students in the
      university)
    \item Change the name of 5000th student
    \begin{itemize}
      \item If it is saved in text file
      \begin{itemize}
        \item Read 4999 lines, skip them and change the 5000th
      \end{itemize}
      \item If it is saved in binary file and each object has the same size
      \begin{itemize}
        \item Jump to the 5000th object by fseek
        \item Why we cannot do the same for text files? More on this later
      \end{itemize}
    \end{itemize}
  \end{itemize}
\end{frame}

\begin{frame}[fragile]
  \frametitle{Moving FPP}
  \begin{minted}[bgcolor=Black]{c}
int fseek(FILE *fp, long offset, int org)
  \end{minted}
  \begin{itemize}
    \item Set FPP in the offset respect to org
    \item org:
    \begin{itemize}
      \item \mintinline{c}|SEEK_SET|: start of file
      \item \mintinline{c}|SEEK_CUR|: current FPP
      \item \mintinline{c}|SEEK_END|: End of file
    \end{itemize}
    \item Returns nonzero if it is unsuccessful
  \end{itemize}
\end{frame}

\begin{frame}[fragile]
  \begin{minted}[bgcolor=Black]{c}
(1,1)(2,2)(3,3)(4,4)(5,5)
  \end{minted}
  \begin{minted}[bgcolor=Black]{c}
fp = fopen("points.bin", "rb");

fread(&p, sizeof(p), 1, fp);
printf("%d %d\n", p.x, p.y); // 1 1

fseek(fp, 2 * sizeof(p), SEEK_SET);
fread(&p, sizeof(p), 1, fp);
printf("%d %d\n", p.x, p.y); // 3 3

fseek(fp, -3 * sizeof(p), SEEK_END);
fread(&p, sizeof(p), 1,fp);
printf("%d %d\n", p.x, p.y); // 3 3

fseek(fp, 1 * sizeof(p), SEEK_CUR);
fread(&p, sizeof(p), 1, fp); // 5 5
printf("%d %d\n", p.x, p.y);
  \end{minted}
\end{frame}

\begin{frame}[fragile]
  \frametitle{Other FPP related functions}
  \begin{itemize}
    \item Find out where is the FPP
    \begin{minted}[bgcolor=Black]{c}
int ftell(FILE *fp)
    \end{minted}
    \item ftell returns the current FPP
    \begin{itemize}
      \item With respect to \mintinline{c}|SEEK_SET|
    \end{itemize}
    \item Reset the FPP to the start of file
    \begin{minted}[bgcolor=Black]{c}
void rewind(FILE *fp)
    \end{minted}
  \end{itemize}
\end{frame}

\begin{frame}[fragile]
  \begin{minted}[bgcolor=Black]{c}
#include <stdio.h>

struct point{
  int x, y;
};

int main(void){
   FILE* fp;

  struct point p;
  int num;

  fp = fopen("points.bin", "rb+");
  if (fp == NULL){
    printf("Cannot read from file\n");
    return -1;
  }

  printf("Enter the number of points: ");
  scanf("%d", &num);

  printf("Enter new X and Y: ");
  scanf("%d %d", &(p.x), &(p.y));

  fseek(fp, (num – 1) * sizeof(p), SEEK_SET);
  fwrite(&p, sizeof(p), 1, fp);

  fclose(fp);

  return 0;
}
  \end{minted}
\end{frame}

\begin{frame}[fragile]
  \frametitle{fseek in Text files}
  \begin{itemize}
    \item Not very useful
    \item Offset counts the number of characters including \mintinline{c}|'\n'|
    \item Typical useful versions
    \begin{itemize}
      \item Go to the start of file
      \begin{minted}[bgcolor=Black]{c}
fseek(fp, 0, SEEK_SET)
      \end{minted}
      \item Go to the end of file
      \begin{minted}[bgcolor=Black]{c}
fseek(fp, 0, SEEK_END)
      \end{minted}
    \end{itemize}
  \end{itemize}
\end{frame}

\section{Bugs and avoiding them}

\begin{frame}
  \frametitle{Common Bugs and Avoiding Them}
  \begin{itemize}
    \item Take care about mode in fopen
    \begin{itemize}
      \item w \& w+: all data in file will be lost
      \item r: you cannot write. fprintf does not do any thing
    \end{itemize}
    \item Take care about text or binary
    \begin{itemize}
      \item fscanf/fprintf don’t do meaningful job in binary files
    \end{itemize}
    \item Check the successful open: fp != NULL
    \item Check EOF as much as possible.
    \item Close the open files.
  \end{itemize}
\end{frame}

\begin{frame}
  \frametitle{Legends}
  \begin{figure}
    \includegraphics[height=.75\textheight]{./img/yoda.jpg}
  \end{figure}
  \pause%
  \centering
  \color{Violet} You
\end{frame}
\begin{frame}
  \begin{itemize}
    \item Thanks for attending this course
  \end{itemize}
\end{frame}

\end{document}
