\documentclass{../c-lecture}

\usepackage{algorithm2e}

\subtitle{Repeating Statements}

\begin{document}

\begin{frame}
  \titlepage{}
\end{frame}
\begin{frame}
  \frametitle{Outline}
  \tableofcontents{}
\end{frame}

\section{Introduction}

\begin{frame}
  \frametitle{Repetition}
  \begin{itemize}
    \item Example: Write a program that read 3 integer and compute average
    \begin{itemize}
      \item It is easy. 3 scanf, an addition, a division and, a printf
    \end{itemize}
    \item Example: Write a program that read 3000 integer and compute average
    \begin{itemize}
      \item ?? 3000 scanf !!!
    \end{itemize}
    \item Example: Write a program that read n integer and compute average
    \begin{itemize}
      \item N??? scanf
    \end{itemize}
  \end{itemize}
\end{frame}

\begin{frame}
  \frametitle{Repetition: counter controlled}
  \begin{itemize}
    \item When we know the number of iteration
    \begin{itemize}
      \item Average of 10 number
    \end{itemize}
  \end{itemize}
  \begin{algorithm}[H]
  \KwData{}
  \KwResult{}
  $counter \gets 0$\;

  \While{$counter < number\_of\_loop\_repetition$}{%
    do something (e.g. read input, take sum)\;
    $counter \gets counter + 1$\;
  }
  \end{algorithm}
\end{frame}

\begin{frame}
  \frametitle{Repetition: sentinel controlled}
  \begin{itemize}
    \item
      When we do \textbf{\color{RubineRed} NOT} know the number of iteration
    \item But we know, when loop terminates
    \begin{itemize}
      \item E.g. Average of arbitrary positive numbers ending with <0
    \end{itemize}
  \end{itemize}
  \begin{algorithm}[H]
  \KwData{}
  \KwResult{}
  $n \gets$ get first input\;

  \While{$n != sentinel$}{%
    do something (sum, \ldots)\;
    $n \gets$ get the next input\;
    \eIf{there is not any valid input}{%
      S1
    }{%
      S2
    }
  }
  \end{algorithm}
\end{frame}

\begin{frame}
  \frametitle{Repetition}
  \begin{itemize}
    \item Repetition is performed by loops
    \begin{itemize}
      \item
        Put all statements to repeat in a \textbf{\color{Orange} loop}
    \end{itemize}
    \item Don’t loop to infinity
    \begin{itemize}
      \item Stop the repetition
      \item Based on some conditions (counter, sentinel)
    \end{itemize}
    \item C has three statements for loops
    \begin{itemize}
      \item \textit{\color{Orange} while} statement
      \item \textit{\color{Orange} do-while} statement
      \item \textit{\color{Orange} for} statement
    \end{itemize}
  \end{itemize}
\end{frame}

\section{while statement}

\begin{frame}
  \frametitle{while statement}
  \mint{c}|while ( <expression> ) <statements>|
\end{frame}

\begin{frame}[fragile]
  \frametitle{print 0 to n}
  \begin{minted}[bgcolor=Black]{c}
#include <stdio.h>

int main(void){
  int n, number;
  number = 0;
  printf("Enter n: ");
  scanf("%d", &n);
  while(number <= n) {
    printf("%d \n", number);
    number++;
  }
  return 0;
}
  \end{minted}
\end{frame}

\section{do-while statement}

\begin{frame}[fragile]
  \frametitle{do-while statement}
  \begin{minted}[bgcolor=Black]{c}
    do {
    <statements>
    } while ( <expression> );
  \end{minted}
\end{frame}

\begin{frame}[fragile]
  \frametitle{sum of the n term: 1/2 + 2/3 ...}
  \begin{minted}[bgcolor=Black]{c}
#include <stdio.h>
#include <stdlib.h>

int main(void) {
  int n;
  double number, sum;

  printf("Enter n > 0: ");
  scanf("%d", &n);

  if(n < 1){
    printf("wrong input");
    return -1;
  }

  sum = 0;
  number = 0.0;
  do {
    number++;
    sum += number / (number + 1.0);
  } while(number &lt; n);

  printf("sum = %lf\n", sum);
  return 0;
}
  \end{minted}
\end{frame}

\section{for statement}

\begin{frame}[fragile]
  \frametitle{for statement}
  \begin{minted}[bgcolor=Black]{c}
    for(<expression1>; <expression2>; <expression3>)
    <statements>
  \end{minted}
\end{frame}

\begin{frame}[fragile]
  \frametitle{N students average grade}
  \begin{minted}[bgcolor=Black]{c}
#include <stdio.h>

int main(void){
  int grade, count, i;
  double average, sum;
  sum = 0;
  printf("Enter the number of students: ");
  scanf("%d", &count);
  for(i = 0; i < count; i++){
    printf("Enter the grade of %d-th student: ", (i + 1));
    scanf("%d", &grade);
    sum += grade;
  }
  average = sum / count;
  printf("The average of your class is %0.3lf\n", average);
  return 0;
}
  \end{minted}
\end{frame}

\begin{frame}[fragile]
  \frametitle{Expressions in for statements}
  \begin{itemize}
    \item
      Expression1 and Expression3 can be any number of expressions, they execute
      in the order

    \begin{minted}[bgcolor=Black]{c}
for(i = 0, j = 0; i < 10; i++, j--)
    \end{minted}
    \item Expression2 at most should be a single expression
    \item
      If multiple expressions then the value of the last one is evaluated as
      True/False

    \begin{minted}[bgcolor=Black]{c}
for(i = 0, j = 0; i &lt; 10, j &gt; -100; i++, j--)
    \end{minted}
  \end{itemize}
\end{frame}

\begin{frame}
  \frametitle{Expressions in for statements}
  \begin{itemize}
    \item Any expression can be empty expression
    \begin{minted}[bgcolor=Black]{c}
for( ; i < 10; i++)
for(;;)
    \end{minted}
  \end{itemize}
\end{frame}

\section{Arrays}

\begin{frame}
  \frametitle{Introduction}
  \begin{itemize}
    \item Algorithms usually work on large data sets
    \begin{itemize}
      \item Sort a set of numbers
      \item Search a specific number in a set of numbers
    \end{itemize}
    \item How to read and store a set of data?
    \item To Read
    \begin{itemize}
      \item Repeat the scanf statement
      \item Use the loop statements
    \end{itemize}
    \item To store the data
    \begin{itemize}
      \item Save each data in a single variable??
      \item 3000 int variables!!!!
    \end{itemize}
  \end{itemize}
\end{frame}

\begin{frame}
  \frametitle{Array}
  \begin{itemize}
    \item
      An \textit{\color{Orange} ordered} collection of
      \textit{\color{Orange} same type} variables

    \item A $1 x n$ vector of
    \begin{itemize}
      \item Integers, chars, floats, ...
    \end{itemize}
    \item Example
    \begin{itemize}
      \item An array of 8 integers
      \begin{table}
      \begin{tabular}{*{8}{c}}
        \toprule

        0 &
        1 &
        2 &
        3 &
        4 &
        5 &
        6 &
        7 \\

        \midrule

        3 &
        1 &
        5 &
        11 &
        10 &
        19 &
        0 &
        12 \\

        \bottomrule
      \end{tabular}
      \end{table}
      \item An array of 5 chars
      \begin{table}
      \begin{tabular}{*{4}{c}}
        \toprule

        0 &
        1 &
        2 &
        3 &
        4 \\

        \midrule

        a &
        z &
        F &
        z &
        k \\

        \bottomrule
      \end{tabular}
      \end{table}
    \end{itemize}
  \end{itemize}
\end{frame}

\begin{frame}
  \frametitle{Arrays in C}
  \begin{itemize}
    \item Array declaration in C
    <p>
      <span class="hl-orange">&lt;Elements’ Type&gt;</span>
      <span class="hl-green">&lt;identifier&gt;</span>[<span class="hl-red"
        >&lt;size&gt;</span
      >]
    </p>
    <p>
      <span class="hl-orange">int</span> <span class="hl-green">arr</span>[<span
        class="hl-red"
        >20</span
      >]
    </p>
    \item
      <span class="hl-orange">&lt;Elements’ Type&gt;</span>: int, char, float,
      \ldots

    \item <span class="hl-red">&lt;size&gt;</span>
    \begin{itemize}
      \item Old compilers (standard): it should be constant
      \item New compilers (standard): it can be variable
    \end{itemize}
    \item Elements in array
    \begin{itemize}
      \item From 0 to (size – 1)
    \end{itemize}
  \end{itemize}
\end{frame}

\begin{frame}
  \frametitle{Example}
  <p>
    <span class="hl-orange">int</span> <span class="hl-green">num</span>[<span
      class="hl-red"
      >20</span
    >]
  </p>
  \begin{itemize}
    \item num is array of 20 <span class="hl-cyan">integers</span>
    \item num[0] is the first integer variable
    \item num[19] is the last integer
  \end{itemize}
  <p>
    <span class="hl-orange">float</span>
    <span class="hl-green">farr</span>[<span class="hl-red">100</span>]
  </p>
  \begin{itemize}
    \item farr is array of <span class="hl-cyan">100</span> floats
    \item farr[0] is the first float
    \item farr[49] is the 50th float
    \item farr[99] is the last float
  \end{itemize}
\end{frame}

\begin{frame}
  \frametitle{Array Initialization: Known Length}
  \begin{itemize}
    <pre><code class="hljs lang-c">int num[3]={10, 20, 60};</code></pre>
    \item num is the array of 3 integers, num[0] is 10, …
    <pre><code class="hljs lang-c">int num[]={40, 50, 60, 70, 70, 80};</code></pre>
    \item num is the array of 6 integers
    <pre><code class="hljs lang-c">int num[10]={40, 50, 60};</code></pre>
    \item num is the array of 10 integers
    \item num[0] is 40, num[1] is 50, num[2] is 60
    \item num[3], num[4], ..., num[9] are 0
  \end{itemize}
\end{frame}

\begin{frame}
  \frametitle{Array Initialization (cont’d)}
  <pre><code class="hljs lang-c">
int num[2]={40, 50, 60, 70};
/* Compile warning */

int num[5]={[0] = 3, [4] = 6};
/* num[5] = {3, 0, 0, 0, 6} */
  </code></pre>
\end{frame}
\begin{frame}
  \frametitle{Initializing Variable Length Arrays}
  <pre><code class="hljs lang-c">
int n;
scanf("%d", &amp;n);
int num[n]={0}; /* Compile error */
  </code></pre>
  <p>Variable length arrays cannot be initialized!</p>
  <pre><code class="hljs lang-c">
for(i = 0; i &lt; n; i++)
  num[i] = 0;
  </code></pre>
\end{frame}
\begin{frame}
  \frametitle{Variable Length Array Declaration}
  <pre><code class="hljs lang-c">
int n;
scanf("%d", &amp;n);
int num[n];
  </code></pre>
  <p><span class="hl-orange">num</span> is usable</p>
  <pre><code class="hljs lang-c">
int n;
int num[n];
scanf("%d", &amp;n);
  </code></pre>
  <p>
    <span class="hl-orange">num</span> is
    <span class="hl-red">not</span> usable, why?
  </p>
\end{frame}
\begin{frame}
  <div class="toc" data-selected="5"></div>
\end{frame}
\begin{frame}
  \frametitle{Empty statements}
  \begin{itemize}
    \item &lt;statement&gt; in loops can be empty
  \end{itemize}
  <pre><code class="hljs lang-c">
while(&lt;expression&gt;) ;
E.g.,
while(i++ &lt;= n) ;

for(&lt;expression1&gt;; &lt;expression2&gt;; &lt;expression3&gt;) ;
E.g.,
for(i = 0; i &lt; 10; printf("%d\n",i), i++) ;
  </code></pre>
\end{frame}
\begin{frame}
  \frametitle{Nested loops}
  \begin{itemize}
    \item &lt;statement&gt; in loops can be loop itself
  \end{itemize}
  <pre><code class="hljs lang-c">
while(&lt;expression0&gt;)
  for(&lt;expression1&gt;; &lt;expression2&gt;; &lt;expression3&gt;)
    &lt;statements&gt;
  </code></pre>
  <pre><code class="hljs lang-c">
for(&lt;expression1&gt;; &lt;expression2&gt;; &lt;expression3&gt;)
  do
    &lt;statements&gt;
  while(&lt;expression&gt;)
  </code></pre>
\end{frame}
\begin{frame}
  \begin{frame}
    \frametitle{Nested loops example}
    \begin{itemize}
      \item A program that takes n and m and prints
      <pre><code class="hljs plaintext">
        *** ….* (m * in each line)
        *** ….*
        …
        *** ….*
        (n lines)
      </code></pre>
    \end{itemize}
  \end{frame}
  \begin{frame}
    <pre><code class="hljs lang-c">
#include &lt;stdio.h&gt;

int main(void){
  int i, j, n, m;
  printf("Enter n &amp; m: ");
  scanf("%d %d", &amp;n, &amp;m);

  for(i = 0; i &lt; n; i++){
    for(j = 0; j &lt; m; j++) {
      printf("*");
    }
    printf("\n");
  }
  return 0;
}
    </code></pre>
  \end{frame}
\end{frame}
\begin{frame}
  \begin{frame}
    \frametitle{Nested loops example}
    \begin{itemize}
      \item A program that takes n and prints
      <pre><code class="hljs plaintext">
        * (i * in i-th line)
        **
        ***
        *** ….*
        (n lines)
      </code></pre>
    \end{itemize}
  \end{frame}
  \begin{frame}
    <pre><code class="hljs lang-c">
#include &lt;stdio.h&gt;

int main(void){
  int i, j, n;
  printf("Enter n: ");
  scanf("%d", &amp;n);

  i = 1;
  while(i &lt;= n) {
    for(j = 0; j &lt; i; j++) {
      printf("*");
    }
    printf("\n");
    i++;
  }
  return 0;
}
    </code></pre>
  \end{frame}
\end{frame}
\begin{frame}
  \frametitle{break statement}
  \begin{itemize}
    \item Exit from loop based on some conditions
  \end{itemize}
  <pre><code class="hljs lang-c">
do{
  scanf("%d", &amp;a);
  scanf("%d", &amp;b);

  if(b == 0)
    break;

  res = a / b;
  printf("a / b = %d\n", res);
} while(b &gt; 0);
  </code></pre>
\end{frame}
\begin{frame}
  \frametitle{continue statement}
  \begin{itemize}
    \item Jump to end of loop and continue repetition
  \end{itemize}
  <pre><code class="hljs lang-c">
do{
  scanf("%f", &amp;a);
  scanf("%f", &amp;b);
  if(b == 0)
    continue;
  res = a / b;
  printf("a / b = %f\n", res);
} while(a &gt; 0);
  </code></pre>
\end{frame}
\begin{frame}
  \frametitle{Which loop?}
  \begin{itemize}
    \item When you know the number of repetition
    \begin{itemize}
      \item Counter-controlled loops
      \item Usually, <span class="hl-orange">for</span> statements
    \end{itemize}
    \item When you don’t know the number of repetitions (sentinel loop)
    \begin{itemize}
      \item Some condition should be check before starting loop
      \begin{itemize}
        \item Usually, <span class="hl-green">while</span> statement
      \end{itemize}
      \item The loop should be executed at least one time
      \begin{itemize}
        \item Usually, <span class="hl-cyan">do-while</span>
      \end{itemize}
    \end{itemize}
  \end{itemize}
\end{frame}
\begin{frame}
  <div class="toc" data-selected="6"></div>
\end{frame}
\begin{frame}
  \frametitle{Common bugs and avoiding them}
  \begin{itemize}
    \item Loop should terminate
    \begin{itemize}
      \item
        E.g., in for loops, after each iteration, we should approach to the stop
        condition

    \end{itemize}
    <pre><code class="hljs lang-c">
for(i = 0; i &lt; 10; i++) //OK
for(i = 0; i &lt; 10; i--) //Bug
    </code></pre>
    \item Initialize loop control variables
    <pre><code class="hljs lang-c">
int i;
for( ; i &lt; 10; i++)
    </code></pre>
  \end{itemize}
\end{frame}
\begin{frame}
  \frametitle{Common bugs and avoiding them}
  \begin{itemize}
    \item Don’t modify for loop controller in loop body
    <pre><code class="hljs lang-c">
for(i = 0; i &lt; 10; i++) {
  ...
  i--; //Bug
}
    </code></pre>
    \item Take care about wrong control conditions
    \begin{itemize}
      \item &lt; vs. &lt;=
      \item = vs. ==
    \end{itemize}
    <pre><code class="hljs lang-c">
int b = 10;
while(a = b){ //it means while(true)
  scanf("%d", &amp;a)
    </code></pre>
  \end{itemize}
\end{frame}

\end{document}
