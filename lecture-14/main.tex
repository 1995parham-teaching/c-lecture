\documentclass{../c-lecture}

\subtitle{Buckle Up}

\usepackage{algorithm2e}

\begin{document}

\begin{frame}
  \titlepage{}
\end{frame}
\begin{frame}
  \frametitle{Outline}
  \tableofcontents{}
\end{frame}

\begin{frame}
  \frametitle{}
  \begin{block}{}
    We bind type modifiers and qualifiers to the left.
  \end{block}
\end{frame}

\begin{frame}
  \frametitle{}
  \begin{block}{}
    We don't use continued declarations.
  \end{block}
\end{frame}

\begin{frame}
  \frametitle{}
  \begin{block}{}
    We use array notation for pointer parameters.
  \end{block}
\end{frame}

\begin{frame}
  \frametitle{}
  \begin{block}{}
    We use function notation for function pointer parameters.
  \end{block}
\end{frame}

\begin{frame}
  \frametitle{}
  \begin{block}{}
    We define variables as close to their first use as possible.
  \end{block}
\end{frame}

\begin{frame}
  \frametitle{}
  \begin{block}{}
    We use prefix notation for code blocks.
  \end{block}
\end{frame}

\end{document}
