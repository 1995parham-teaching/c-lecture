\documentclass{../c-lecture}

\subtitle{Making Decisions}

\begin{document}

\begin{frame}
  \titlepage{}
\end{frame}
\begin{frame}
  \frametitle{Outline}
  \tableofcontents{}
\end{frame}

\section{Conditions and Boolean operations}

\begin{frame}
  \frametitle{Decision}
  \begin{itemize}
    \item Decisions are based on conditions
    \begin{itemize}
      \item If it is snowing
        then We will cancel the game
      \item If the class is not canceled
        then I will attend
        else I will go to gym
    \end{itemize}
    \item In programming
    \begin{itemize}
      \item Do statements based on conditions
      \begin{itemize}
        \item \textbf{\color{LimeGreen} True}: The statements will be done
        \item \textbf{\color{RubineRed} False}: The statements won't be done
      \end{itemize}
    \end{itemize}
  \end{itemize}
\end{frame}

\begin{frame}
  \frametitle{Conditions}
  \begin{itemize}
    \item Conditions by comparisons; e.g.,
    \begin{itemize}
      \item Weather vs. snowing
      \item Variable x vs. a value
    \end{itemize}
    \item Comparing numbers: Relational Operators
  \end{itemize}
\end{frame}

\begin{frame}
  \frametitle{Conditions}
  <table>
    <thead>
      <th>Relational Operator</th>
      <th>Meaning</th>
    </thead>
    <tbody>
      <tr>
        <td>&lt;</td>
        <td>is less than</td>
      </tr>
      <tr>
        <td>&lt;=</td>
        <td>is less than or equal to</td>
      </tr>
      <tr>
        <td>&gt;</td>
        <td>is greater than</td>
      </tr>
      <tr>
        <td>&gt;=</td>
        <td>is greater than or equal to</td>
      </tr>
      <tr>
        <td>==</td>
        <td>is equal to</td>
      </tr>
      <tr>
        <td>!=</td>
        <td>is not equal to</td>
      </tr>
    </tbody>
  </table>
\end{frame}

\begin{frame}[fragile]
  \frametitle{Relations}
  \begin{itemize}
    \item Relations are not a complete statement
    \begin{minted}[bgcolor=Black]{c}
int a, b;
a == b; // statement with no effect
a <= b; // statement with no effect
    \end{minted}
    \item Relations produce a boolean value
    \begin{minted}[bgcolor=Black]{c}
int a, b;
bool bl; // #include <stdbool.h>
bl = a == b;
bl = a <= b;
    \end{minted}
  \end{itemize}
\end{frame}
\begin{frame}
  \frametitle{Boolean operations}
  \begin{itemize}
    \item Multiple conditions in decision making
    \item Logical relation between conditions
    \begin{itemize}
      \item
        \textmd{\color{LimeGreen} if} you are student
        \textmd{\color{YellowOrange} and} you have the programming course
        \textmd{\color{Turquoise} then} You should read the book
    \end{itemize}
  \end{itemize}
\end{frame}
\begin{frame}
  \frametitle{Boolean operations}
  \begin{itemize}
    \item and \&\&
    \item or ||
    \item not !
  \end{itemize}
  <table>
    <thead>
      <th>p</th>
      <th>q</th>
      <th>p &amp;&amp; q</th>
      <th>p || q</th>
      <th>!p</th>
    </thead>
    <tbody>
      <tr>
        <td>False</td>
        <td>False</td>
        <td>False</td>
        <td>False</td>
        <td>True</td>
      </tr>
      <tr>
        <td>False</td>
        <td>True</td>
        <td>False</td>
        <td>True</td>
        <td>True</td>
      </tr>
      <tr>
        <td>True</td>
        <td>False</td>
        <td>False</td>
        <td>True</td>
        <td>False</td>
      </tr>
      <tr>
        <td>True</td>
        <td>True</td>
        <td>True</td>
        <td>True</td>
        <td>False</td>
      </tr>
    </tbody>
  </table>
\end{frame}

\begin{frame}
  \frametitle{Boolean operations (cont’d)}
  \begin{itemize}
    \item Examples
  \end{itemize}
  <pre><code class="hljs lang-c">
bool a = true, b = false, c;
c = !a; // c = false
c = a &amp;&amp; b; // c = false
c = a || b; // c = true
c = !a || b; // c = false
  </code></pre>
\end{frame}
\begin{frame}
  \frametitle{Precedence}
  <img src="img/c-precedence.png" alt="c-precedence" />
\end{frame}
\begin{frame}
  \frametitle{Relations, No type effect}
  \begin{itemize}
    \item
      Usual arithmetic conversions are performed on both operands following the
      rules for arithmetic operators.

    \item The values are compared after conversions.
  \end{itemize}
  <pre><code class="hljs lang-c">
int a = 10, b = 20;
float f = 54.677f;
double d = 547.775;
char c1 = 'A', c2 = 'a';
bool b1;

b1 = a == f; // false
b1 = a &lt;= d + 5; // true
b1 = d &lt; c1 * 10; // true
b1 = c1 == c2; // false
b1 = '1' &lt; '2'; // true
b1 = c1 + f &lt; d + a; // true
  </code></pre>
\end{frame}
\begin{frame}
  \frametitle{Casting}
  \begin{itemize}
    \item In logical operations
    \begin{itemize}
      \item 0: False
      \item non-zero: True
    \end{itemize}
    \item In mathematical &amp; comparison operations
    \begin{itemize}
      \item False: 0
      \item True: 1
    \end{itemize}
  \end{itemize}
  <pre><code class="hljs lang-c">
bool b1, b2;
int i = 0, j = 20;
b1 = i &amp;&amp; j; // b1 = false
b2 = j || j; // b2 = true
i = b1 + b2; // i = 1
j = (i &lt; j) + (b1 &amp;&amp; b2); // j = 1
  </code></pre>
\end{frame}
\begin{frame}
  \begin{frame}
    \frametitle{Examples}
    \begin{itemize}
      \item x in [10, 20]
      \item <span class="hl-red">Wrong Version</span>
      \begin{itemize}
        \item 10 &lt;= x &lt;= 20
        \item Let x = 30
        <ol>
          \item 10 &lt;= 30 &lt;= 20
          \item (10 &lt;= 30) &lt;= 20
          \item true &lt;= 20
          \item 1 &lt; 20
          \item True
        </ol>
      \end{itemize}
    \end{itemize}
  \end{frame}
  \begin{frame}
    \frametitle{Examples}
    \begin{itemize}
      \item x in [10, 20]
      \item <span class="hl-green">Correct Version</span>
      \begin{itemize}
        \item (10 &lt;= x) &amp;&amp; (x &lt;= 20)
        \item Let x = 30
        <ol>
          \item (10 &lt;= 30) &amp;&amp; (30 &lt;= 20)
          \item true &amp;&amp; false
          \item false
        </ol>
      \end{itemize}
    \end{itemize}
  \end{frame}
\end{frame}
\begin{frame}
  \begin{frame}
    \frametitle{Examples}
    \begin{itemize}
      \item a, b &gt; 0
      \item <span class="hl-red">Wrong Version</span>
      \begin{itemize}
        \item a &amp;&amp; b &gt; 0
        \item Let a = -10, b = 20
        <ol>
          \item -10 &amp;&amp; 20 &gt; 0
          \item -10 &amp;&amp; (20 &gt; 0)
          \item -10 &amp;&amp; true
          \item true &amp;&amp; true
          \item true
        </ol>
      \end{itemize}
    \end{itemize}
  \end{frame}
  \begin{frame}
    \frametitle{Examples}
    \begin{itemize}
      \item a, b &gt; 0
      \item <span class="hl-green">Correct Version</span>
      \begin{itemize}
        \item (a &gt; 0) &amp;&amp; (b &gt; 0)
        \item Let a = -10, b = 20
        <ol>
          \item (-10 &gt; 0) &amp;&amp; (20 &gt; 0)
          \item false &amp;&amp; true
          \item false
        </ol>
      \end{itemize}
    \end{itemize}
  \end{frame}
\end{frame}
\begin{frame}
  <div class="toc" data-selected="1"></div>
\end{frame}
\begin{frame}
  \frametitle{Type of statements}
  \begin{itemize}
    \item Expression statement
    \begin{itemize}
      \item Single statements
    \end{itemize}
    <pre><code class="hljs lang-c">
x = y + 10;
scanf("%d", &amp;i);
    </code></pre>
    \item Control statement
    \begin{itemize}
      \item Control the flow of program
      \item Decisions and loops
    \end{itemize}
    \item Compound statement
    \begin{itemize}
      \item Starts with { and ends with }
      \item All statements can be between { and }
    \end{itemize}
  \end{itemize}
\end{frame}
\begin{frame}
  \frametitle{if statement}
  \begin{itemize}
    \item Decision making in C
    <pre><code class="hljs lang-c">
if( &lt;expression&gt; )
  &lt;statements1&gt;
else
  &lt;statements2&gt;
    </code></pre>
    \item Expression
    \begin{itemize}
      \item A boolean statement: <span class="hl-green">a &lt;= b</span>
      \item
        A mathematical statement: <span class="hl-green">a + b</span> or a
        variable: <span class="hl-green">a</span>

      \begin{itemize}
        \item zero &rarr; false
        \item non-zero &rarr; true
      \end{itemize}
    \end{itemize}
  \end{itemize}
\end{frame}
\begin{frame}
  \frametitle{Even-Odd}
  <pre><code class="hljs lang-c">
#include &lt;stdio.h&gt;
int main(void){
  int number_to_test, remainder;
  printf("Enter your number to be tested: ");
  scanf("%d", &amp;number_to_test);
  remainder = number_to_test % 2;
  if(remainder == 0)
    printf ("The number is even.\n");
  else
    printf ("The number is odd.\n");
  return 0;
}
  </code></pre>
\end{frame}
\begin{frame}
  \frametitle{Statements in if-else}
  \begin{itemize}
    \item Empty statement
    <pre><code class="hljs lang-c">
if(a &gt; b)
  printf("A is larger \n");
else
  ;
    </code></pre>
    \item Block statements
    <pre><code class="hljs lang-c">
if(a &lt;= b){
  printf("A is less than b or ");
  printf("A is equal b\n");
}
else
  printf("A is greater than b\n");
    </code></pre>
  \end{itemize}
\end{frame}
\begin{frame}
  <pre><code class="hljs lang-c">
#include &lt;stdio.h&gt;
int main(void){
  int i;
  char c;

  printf("Enter a char: ");
  scanf(" %c", &amp;c);

  printf("Enter an int: ");
  scanf("%d", &amp;i);

  if(i &gt; 0)
    printf("Your number is larger than 0\n");
  else
    printf("Your number is less than or equal 0\n");

  if((c &gt;= '0') &amp;&amp; (c &lt;= '9'))
    printf("Your char is Numeric \n");

  return 0;
}
  </code></pre>
\end{frame}
\begin{frame}
  \frametitle{More than two choices}
  \begin{itemize}
    \item If statement: 2 choices
    \begin{itemize}
      \item If conditions are true &rarr; if statements
      \item If conditions are false &rarr; else statements
    \end{itemize}
    \item How to make decisions when there are multiple choices?
  \end{itemize}
\end{frame}
\begin{frame}
  \frametitle{Map numeric grade to alphabetic}
  <pre><code class="hljs lang-c">
int numg;
char alphag;
if(numg &lt; 25)
  alphag = 'D';
if((numg &gt;= 25) &amp;&amp; (numg &lt; 50))
  alphag = 'C';
if((numg &gt;= 50) &amp;&amp; (numg &lt; 75))
  alphag = 'B';
if(numg &gt;= 75)
  alphag = 'A';
  </code></pre>
\end{frame}
\begin{frame}
  \frametitle{More than two choices}
  \begin{itemize}
    \item To avoid repeating conditions in if statements
    \item To avoid running unnecessary statements
    \item <span class="hl-red">Nested</span> if: check multiple conditions
    \begin{itemize}
      \item &lt;Statements 1&gt; becomes an if-else statement
      \item &lt;Statements 2&gt; becomes an if-else statement
      \item Repeat it as many as needed
    \end{itemize}
  \end{itemize}
\end{frame}
\begin{frame}
  \frametitle{Nested if-else}
  <pre><code class="hljs lang-c">
if(c1 &amp;&amp; c2)
  s1
if(c1 &amp;&amp; !(c2))
  s2
if(!(c1) &amp;&amp; c3)
  s3
if(!(c1) &amp;&amp; !(c3))
  s4
  </code></pre>
  <pre><code class="hljs lang-c">
if(c1)
  if(c2)
    s1
  else
    s2
else
  if(c3)
    s3
  else
    s4
  </code></pre>
\end{frame}
\begin{frame}
  \frametitle{Map numeric grade to alphabetic}
  <pre><code class="hljs lang-c">
int numg;
char alphag;

if (numg &lt; 25)
  alphag = ‘D’;
else {
  if (numg &lt; 50)
    alphag = ‘C’;
  else {
    if (numg &lt; 75)
      alphag = ‘B’;
    else
      alphag = ‘A’;
  }
}
  </code></pre>
\end{frame}
\begin{frame}
  \frametitle{Nested if-else}
  <pre><code class="hljs lang-c">
if (&lt;condition 1&gt;)
  &lt;statement 1&gt;
else {
  if(&lt;condition 2&gt;)
    &lt;statement 2&gt;
  else
    &lt;statement 3&gt;
}
  </code></pre>
  <pre><code class="hljs lang-c">
if (&lt;condition 1&gt;)
  &lt;statement 1&gt;
else if (&lt;condition 2&gt;)
  &lt;statement 2&gt;
else
  &lt;statement 3&gt;
  </code></pre>
\end{frame}
\begin{frame}
  \frametitle{Map numeric grade to alphabetic}
  <pre><code class="hljs lang-c">
int numg;
char alphag;

if(numg &lt; 25)
  alphag = 'D';
else if(numg &lt; 50)
  alphag = 'C';
else if(numg &lt; 75)
  alphag = 'B';
else
  alphag = 'A';
  </code></pre>
\end{frame}
\begin{frame}
  \frametitle{Nested if: Incomplete branch}
  <ol>
    \item <span class="hl-orange">else</span> part is optional
    \item
      <span class="hl-orange">else</span> always associates with the
      <span class="hl-green">nearest</span>
      <span class="hl-orange">if</span>

    \item 1 + 2 can be dangerous specially in incomplete branchs
  </ol>
  <p>Example: Tell user to move or game over</p>
  <pre><code class="hljs lang-c">
if(gameIsOver == 0)
  if(playerToMove == YOU)
    printf ("Your Move\n");
else
  printf ("The game is over\n");
  </code></pre>
  \begin{itemize}
    \item To avoid error you should
    \item Close off you code or Use Empty statements
  \end{itemize}
\end{frame}
\begin{frame}
  <div class="toc" data-selected="2"></div>
\end{frame}
\begin{frame}
  \frametitle{switch-case: Multiple choices}
  \begin{itemize}
    \item Multiple conditions
    \begin{itemize}
      \item if-else if-else if-….
    \end{itemize}
    \item
      Select from alternative <span class="hl-orange">values</span> of a
      <span class="hl-green">variable</span>

    \begin{itemize}
      \item switch-case
      \item
        Values should be <span class="hl-orange">constant</span> not expression:
        i, i+j,

      \item
        Values &amp; Variables should be <span class="hl-orange">int</span> or
        <span class="hl-orange">char</span>

    \end{itemize}
  \end{itemize}
  <pre><code class="hljs lang-c">
switch (variable) {
  case value1:
    &lt;statements 1&gt;
  case value2:
    &lt;statements 2&gt;
}
  </code></pre>
\end{frame}
\begin{frame}
  \frametitle{How does switch-case work?}
  \begin{itemize}
    \item Each switch-case can be rewritten by if-else
  \end{itemize}
  <pre><code class="hljs lang-c">
if(variable == value1)}
  &lt;statements 1&gt;
  &lt;statements 2&gt;
}
else if(variable == value2){
  &lt;statements 2&gt;
}
  </code></pre>
\end{frame}
\begin{frame}
  \frametitle{switch-case: complete version}
  <pre><code class="hljs lang-c">
switch(variable) {
  case value1:
    &lt;statements 1&gt;
    break;
  case value2:
    &lt;statements 2&gt;
    break;
  default:
    &lt;statements 3&gt;
}
  </code></pre>
  <pre><code class="hljs lang-c">
if(variable == value1) {
  &lt;statements 1&gt;
}
else if(variable == value2) {
  &lt;statements 2&gt;
}
else {
  &lt;statements 3&gt;
}
  </code></pre>
\end{frame}
\begin{frame}
  \frametitle{switch-case (cont’d)}
  \begin{itemize}
    \item All values used in case should be different
  \end{itemize}
  <pre><code class="hljs lang-c">
switch(i){ //Error
case 1:
…
case 2:
…
case 1:
  </code></pre>
\end{frame}
\begin{frame}
  \frametitle{switch-case (cont’d)}
  \begin{itemize}
    \item All values must be value, not expression of variables
  \end{itemize}
  <pre><code class="hljs lang-c">
switch(i){ //Error
case j:
…
case 2:
…
case k+10:
  </code></pre>
\end{frame}
\begin{frame}
  \frametitle{switch-case: multiple matches}
  <pre><code class="hljs lang-c">
switch(variable) {
  case value1:
  case value2:
    &lt;statements 1&gt;
    break;
  case value3:
    &lt;statements 2&gt;
}
  </code></pre>
  <pre><code class="hljs lang-c">
if(
  (variable == value1) ||
  (variable == value2)
){
    &lt;statements 1&gt;
} else if (variable == value3) {
    &lt;statements 2&gt;
}
  </code></pre>
\end{frame}
\begin{frame}
  \frametitle{switch-case vs. if-else}
  \begin{itemize}
    \item
      <span class="hl-orange">if-else</span> is more powerful than
      <span class="hl-orange">switch-case</span>

    \item
      <span class="hl-orange">switch-case</span> is only for checking the
      <span class="hl-green">values of a variable</span> and the values must be
      <span class="hl-green">constant</span>

    \item if-else is more suitable in some cases , e.g.,
    <pre><code class="hljs lang-c">
double var1, var2;
if(var1 &lt;= 1.1)
  &lt;statements 1&gt;
if(var1 == var2)
  &lt;statements 2&gt;
    </code></pre>
  \end{itemize}
\end{frame}
\begin{frame}
  \frametitle{Nested switch-case}
  <pre><code class="hljs lang-c">
bool b; //b = x &amp;&amp; y
switch (x){
  case 0:
    b = 0;
    break;
  case 1:
    switch(y){
      case 0:
        b = 0;
        break;
      case 1:
        b = 1;
        break;
    }
    break;
}
  </code></pre>
\end{frame}
\begin{frame}
  <div class="toc" data-selected="3"></div>
\end{frame}
\begin{frame}
  \frametitle{Conditional Expression}
  \begin{itemize}
    \item Assign value according to conditions
    \item A ternary operator
  \end{itemize}
  <pre><code class="hljs lang-c">
int i, j, k;
bool b;

i = b ? j : k;
  </code></pre>
  <pre><code class="hljs lang-c">
if (b)
  i = j;
else
  i = k;
  </code></pre>
\end{frame}
\begin{frame}
  \frametitle{Conditional Expression: Examples}
  <pre><code class="hljs lang-c">
y = (x &gt; 0) ? x : -x

sign = (x &lt; 0) ? -1 : ((x &gt; 0) ? 1 : 0)
  </code></pre>
\end{frame}
\begin{frame}
  <div class="toc" data-selected="4"></div>
\end{frame}
\begin{frame}
  \begin{frame}
    \frametitle{Common Bugs}
    \begin{itemize}
      \item Equality of floating point numbers
      \item
        Two float numbers may or may <span class="hl-red">NOT</span> be equal

    \end{itemize}
    <pre><code class="hljs lang-c">
double d1, d2;
d1 = 1e20 + 1;
d2 = 1e20 - 1;
if(d1 == d2)
  printf("They are equal :-o \n");
else
  printf("They are not equal :D \n");
    </code></pre>
    <p><span class="hl-red">They are equal :-o</span></p>
  \end{frame}
  \begin{frame}
    <p>
      Instead of checking for equality, check their difference to be in a
      small-enough period.
    </p>
    <pre><code class="hljs lang-c">
double d1, d2;
d1 = 1e20 + 1;
d2 = 1e20 - 1;
if(fabs(d1 - d2) &lt;= 0.0000001)
  printf("They are equal :-o \n");
else
  printf("They are not equal :D \n");
    </code></pre>
  \end{frame}
\end{frame}
\begin{frame}
  \frametitle{Common Bugs}
  \begin{itemize}
    \item Danger of empty statement
    \item Danger of assignment (=) and equality (==)
    <pre><code class="hljs lang-c">
int a = 10;
int b = 20;
if (a = b) // logical but not compile error!!!
    </code></pre>
    \item Danger of similarity between C and mathematic
    <pre><code class="hljs lang-c">
if (a &lt; b &lt; c) // Logical Error
if (a &amp;&amp; b &gt; 0) // Logical Error
    </code></pre>
  \end{itemize}
\end{frame}
\begin{frame}
  \frametitle{Avoiding Bugs}
  \begin{itemize}
    \item Precedence of operators
    \item Use parenthesis in conditions
    \item Close-off code as much as you can
  \end{itemize}
\end{frame}
\begin{frame}
  \frametitle{Debugging by assert}
  \begin{itemize}
    \item
      The assert macro is defined in <span class="hl-orange">assert.h</span>

    \item assert(an expression)
    \begin{itemize}
      \item If the expression is true &rarr; nothing
      \item If the expression is false &rarr; error message + halt
    \end{itemize}
  \end{itemize}
\end{frame}

\end{document}
